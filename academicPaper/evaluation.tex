\section{Experiments and Guidelines}
\label{sec:evaluation}

To validate the concept of viewport-adaptive 360-degree video streaming, we have developed
a software tool, which provides the main features that the system should offer at both client and 
server side. This tool allowed us to run some first experiments, which we report hereafter, and to
enunciate a few guidelines for the design of these systems.

We first describe the main features of our software tool to better explain our experiments.
Since the code and all the details are publicly available,%
\footnote{\url{https://github.com/xmar/360Transformations}} the software can be used and enhanced
by the scientific community to make further studies and to develop real systems. The main
features include: $(i)$ \emph{Projection from a spherical video into any of the four
geometric layouts and vice versa}. The spherical video is the pivot format from which it
is possible to project in any layout.
Note that the majority of 360-degree videos that are currently available are encoded and stored after
an equirectangular projection, but our tools enables re-projecting these videos
in another layout without information loss. $(ii)$ \emph{Adjusting video qualities for each 
geometric face of any
layout}. We set the resolution of the face in number of
pixels and the encoding bitrate goal. Each face is encoded into its own video
track to get different
bitrate goal per face. And $(iii)$ \emph{Viewport extraction for any \ac{FoV} center in the
sphere}. This feature includes the management of different video bit-rate and resolutions
when the extracted viewport overlaps several faces with different qualities.

In the following, we report the experiment of measuring the video quality of viewports, which
are extracted from 360-degree videos that are projected onto various geometric layouts and 
with various
face quality arrangements. We evaluate video quality against two reference videos.
\begin{itemize}
\item \emph{The original video at full quality}. We used the
``\emph{Bridge
jumping}''\footnote{\url{https://youtu.be/yarcdW91djQ}} video, which is a 4k equirectangular
video, from which it is possible to extract viewports at 1080p resolution. This original video 
enables the use of objective video quality assessment tools such as \ac{MS-SSIM} and
\ac{PSNR}. Since we compare several encoded versions of the same video against an original
one, these tools provide
a fair performance evaluation.
\item \emph{The same video re-encoded at a target bit-rate}. Recall that our ultimate motivation 
is to reduce the bit-rate of the delivered video, while maintaining a high video quality. We set
a bit-rate target, which corresponds to half of the bit-rate of the original video. With this bit-rate
\emph{budget}, our competitor is the original equirectangular video re-encoded with 
\ac{HEVC} by specifying this bit-rate target. We call it \emph{uniEqui} to state that, in this
video, the quality is uniform. 
\end{itemize}

\subsection{Geometric Layout and Face Quality}
\label{sec:geometry}
The preparation of the 360-degree video in our viewport-adaptive streaming system
requires several settings. We distinguish some global settings (the number of \acp{QEC} ($n$), 
the number of representations ($k$), the segment length ($x$) and the geometric layout) 
and settings \emph{per representation}
(the target bit-rate, the number of different qualities in the representation, the quality 
arrangement onto the different
faces). Our tool enables testing the numerous configurations regarding the latter. We do not aim
at being exhaustive in this paper. The study of the best configurations and the development
of better video encoding tools is left for future works within the signal processing 
community. We restrict here our study on the impact of the geometric layout.

A \textit{quality-differentiated layout} is the combination of a geometric layout and video quality 
arrangement onto the geometric faces. The performance of the layout can be studied with
regards to two scales: \emph{the best viewport quality}, which is the quality of the extracted viewport 
when the FoV center and the QEC perfectly matches, and the \emph{sensitivity to head movements}, 
which is the degradation of the viewport quality when the distance between the FoV center 
and the QEC increases. 
To evaluate both scales, we chose one origin point \ac{QEC} in the spherical video. And then we iterated
over the orthodromic distance $d$, which increases from 0 to $\pi$ (the furthest point). For each
value $d$, we randomly picked ten \ac{FoV} centers at the distance $d$ from the origin \ac{QEC}
and we extracted the viewport, which we compared to the same viewport extracted from the 
full-quality original video to get an objective video quality measure.

% ====
%the \ac{QoE} degradation
%depending on the distance between a \ac{FoV} center and a
%\ac{QEC}. We consider the video is projected on the unit sphere. The
%distance is measured using the orthodromic distance on this sphere. We
%used a distance step of 0.25 units. For each distance $d$, for each
%layout, and for each \ac{QEC} we randomly select $10$ \ac{FoV} center
%located at the distance $d$ from the \ac{QEC} and we extract the
%associated \acp{FoV}. Then we use the \ac{FoV} at the same position
%extracted from the original video to compute the \ac{MS-SSIM} and the
%\ac{PSNR} of each \ac{FoV}.
%
%The Figure~\ref{fig:dist_quality} displays the average \ac{MS-SSIM} of
%the \acp{FoV} depending on their distance from the different \acp{QEC}
%and Figure~\ref{fig:dist_quality_psnr} represents the average
%\ac{PSNR} gap from the equirectagular layout encoded with the same
%bitrate budget. Both Figures are from the same simulation results. We
%observe that the cube map layout generates the best results. The
%\ac{QoE} from the pyramid layout decreases quickly: After a distance of
%$1$ the quality is under the default equirectangular layout quality.
%The quality of the cube map layout is never under the
%quality of the default equirectangular layout. This was not expected. It may
%be because we used tiles inside the equirectangular layout, this avoid
%the encoder to efficiently compress the video. \noteXC{There is still
%some simulation running to see if we can show better results}

\begin{figure}
    \pgfplotscreateplotcyclelist{My color list}{%
    {gray,solid, very thick},%
    {black!25, dashed, very thick},%
    {black!75, densely dashed, very thick},%
    {black, densely dotted, very thick},%
}

\pgfplotsset{every axis legend/.append style={
        at={(0,0.97)},
        anchor=south west,
        draw=none,
        fill=none,
        legend columns=4,
        column sep=5pt,
        /tikz/every odd column/.append style={column sep=0cm},
        font=\scriptsize
}}
%\pgfplotsset{grid style={dashed,gray}}
\pgfplotsset{minor grid style={gray!20}}
\pgfplotsset{major grid style={dotted}}


\begin{tikzpicture}
    \begin{axis}[
            ylabel={PSNR gap (in dB)},
            xlabel={Orthodromic distance (in distance units)},
            width=1.05\linewidth,
            height=0.5\linewidth,
            cycle list name={My color list},
			 extra y ticks={0},
            extra y tick labels={\textit{\vphantom{lj}uniEqui}},
            extra y tick style={grid=major, 
               tick pos=left, 
               ticklabel pos=left, tick
               align=inside,
               grid=minor,
               tick label style={rounded corners,
                  fill=gray!20,
                  font=\tiny, 
                  inner sep=2pt,
                  anchor=west}},
            legend cell align=left,
            xmin = 0,
            xmax = 3.14,
%            ymajorgrids,
        ]

        \addplot+ table [x=distance,
        y=qualityEquirectangularTiledLower]{results/distanceQuality_psnr.csv};
        \addplot+ table [x=distance,
        y=qualityCubeMapLower]{results/distanceQuality_psnr.csv};
        \addplot+ table [x=distance,
        y=qualityPyramidLower]{results/distanceQuality_psnr.csv};
        \addplot+ table [x=distance,
        y=qualityRhombicDodeca]{results/distanceQuality_psnr.csv};
        \legend{Equirec,CubeMap,Pyramid,Dodeca}

    \end{axis}
\end{tikzpicture}

    \caption{Average \acs{PSNR} gap compared to the \emph{uniEqui} layout, depending on the distance from the \acs{QEC}. Global bitrate budget \SI{6}{\mega bps}}
    \label{fig:dist_quality_psnr}
\end{figure}

We represent in Figure~\ref{fig:dist_quality_psnr} the difference of \ac{PSNR} between
the extracted viewport from our quality-differentiated layouts and the same viewport extracted from
the reference \textit{uniEqui} layout. For each geometric layout (equirectangular 
panorama with $8\times 8$ tiles, cube map, pyramid, and dodecahedron), we have tested 
numerous quality arrangements with respect to the overall bit-rate budget. We selected
here the ``best" arrangement for each layout.

The projection on a cube map appears to be the best choice for the content provider. The quality of
the viewport when QEC and FoV center matches ($d=0$) is 3dB better than the viewport extracted
from the reference uniformly encoded 360-degree video at the same bit-rate. The gain in quality 
decreases when the distance $d$ increases but the quality of the cube map layout is always greater 
than the other layouts. Note that the pyramid projection is more sensitive to head movements than
the other layouts, due to the projection under-sampling, with a video quality that is worse than
the standard \emph{uniEqui} for distance greater than $1.2$. The choice of Facebook to 
use the pyramid
layout~\cite{facebook} would thus require cutting the video into shorter segments 
(duration \SI{1}{\second}) to maintain a better \ac{QoE} than the standard layout.

%\begin{figure}
%    \pgfplotscreateplotcyclelist{My color list}{%
    {gray,solid, very thick},%
    {black!25, dashed, very thick},%
    {black!75, densely dashed, very thick},%
    {black, densely dotted, very thick},%
}

\pgfplotsset{every axis legend/.append style={
        at={(0,0.97)},
anchor=south west,
draw=none,
fill=none,
legend columns=-1,
column sep=15pt,
/tikz/every odd column/.append style={column sep=0cm},
%font=\tiny
}}
\pgfplotsset{grid style={dashed,gray}}
\pgfplotsset{minor grid style={dotted,red!20!gray}}
\pgfplotsset{major grid style={dotted,green!50!black}}


\begin{tikzpicture}
    \begin{axis}[
            ylabel={MS-SSIM},
            xlabel={Orthodromic distance},
            width=1.05\linewidth,
            height=0.5\linewidth,
 %           unit vector ratio=1 2 1,
 %           x=0.4cm,
            cycle list name={My color list},
            %ytick={0,2,4,...,10},
            %xtick pos=left,
            %x tick label style={at={(current axis.right of origin)},rotate=45,anchor=north east},
            %minor y tick num={1},
            legend cell align=left,
            xmin = 0,
            xmax = 3.14,
            %ymin = 0.7,
            ymax = 1,
            ymajorgrids,
            yminorgrids,
%            axis on top,
            %y filter/.code={\pgfmathparse{#1/1000000}\pgfmathresult},
        ]

        \addplot+ table [x=distance, y=qualityEquirectangularTiled]{results/distanceQuality.csv};
        \addplot+ table [x=distance, y=qualityCubMap]{results/distanceQuality.csv};
        \addplot+ table [x=distance, y=qualityPyramidal]{results/distanceQuality.csv};
        \addplot+ table [x=distance, y=qualityRhombicDodeca]{results/distanceQuality.csv};
        \addplot+ table [x=distance, y=qualityAverageEquiTi]{results/distanceQuality.csv};
        \legend{EqTiled,CubeMap,Pyramid,RhombicDodeca,AverageEqui}

    \end{axis}
\end{tikzpicture}

%    \caption{Average \acs{MS-SSIM} depending on the distance from the
%    \acs{QEC}. Bit-rate budget: \SI{6}{\mega bps}}
%    \label{fig:dist_quality}
%\end{figure}

\subsection{Dynamic Viewport Quality}
\label{sec:qoe}