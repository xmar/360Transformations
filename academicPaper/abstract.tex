\begin{abstract}
The delivery and display of 360-degree videos on Head-Mounted Displays (HMDs) presents 
many technical challenges. 360-degree videos are ultra high resolution spherical videos, which 
contain an omnidirectional view of the scene. However only a portion of this scene is displayed 
on the HMD. Moreover, HMD need to respond in 10 ms to head movements, which prevents the 
server to send only the displayed video part based on client feedback. To reduce the bandwidth 
waste, while still providing an immersive experience, a viewport-adaptive 360-degree video 
streaming system is proposed. The server prepares multiple video representations, which differ not 
only by their bit-rate, but also by the qualities of different scene regions. The client chooses a 
representation for the next segment such that its bit-rate fits the available throughput and a full 
quality region matches its viewing. We investigate the impact of various spherical-to-plane 
projections and quality arrangements on the video quality displayed to the user, showing that the 
cube map layout offers the best quality for the given bit-rate budget. An evaluation with a dataset 
of users navigating 360-degree videos demonstrates that segments need to be short enough to 
enable frequent view switches.
\end{abstract}
%%% Local Variables:
%%% mode: latex
%%% TeX-master: "paper"
%%% End:
