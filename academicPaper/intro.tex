\section{Introduction}
\label{sec:introduction}

\subsection{Context and Motivations}

The popularity of omnidirectional multimedia systems (also known as interactive 360-degree video) 
has grown with the advent of both new capturing systems
(including catadioptric optical systems and multi-camera with stitching systems) and new video
consumption systems (including head-mounted display and interactive HTML5 video players).
However, the delivery of 360-degree multimedia content, from the servers of the content providers
to the end-users,
is still a challenge. Head-mounted devices display two videos (one per
eye), each of them with a high resolution (typically $1080\times 1200$ for state-of-the-art devices)
and a high frame rate. Both videos, which correspond to the \ac{FoV} on both eyes, are extracted
from a wider spherical video with a resolution that is three to four times larger.

None of the current solutions for the delivery of 360-degree videos is entirely satisfactory. Sending only 
the  \ac{FoV} videos is the implementation that is the least bandwidth-hungry. It requires however the server to 
compute the \ac{FoV} from the spherical video for each end-user. Moreover, it does not enable fast
navigation within the scene: When the client changes the 
\ac{FoV}, the notification to the server and the delivery of the newly adjusted \ac{FoV} videos 
introduce delays. Another delivery implementation is to send the full spherical video and to let the device
extract the \ac{FoV} videos. This solution enables fast navigation but the bandwidth requirements are 
significant.

We explore in this paper a solution where the server offers multiple versions of the same 360-degree 
video. Each version contains the full spherical video with an emphasis on a given angle of vision, \textit{i.e.},
the part of the video that is in front of the angle of vision is at the highest quality and the video quality degrades
for the other parts. The client device chooses the video version according to the eye attention such that the 
the front view of the video version is the closest to the \ac{FoV} videos. When the end-user changes the 
\ac{FoV}, the client switches to another video version, but the movement is smooth because the device 
can always extract \ac{FoV} videos from any video version, even if it is at a lower quality until 
it gets the video version that best suits the eye attention.

\subsection{Limitations of Previous Work}

The principle of differentiated video quality in the delivery of omnidirectional videos has been sketched 
in a recent short paper by~\citet{ochi_live_2015}. Their proposal is based on the idea of video
tiling, which has been proposed for panorama 
video~\cite{sanchez_compressed_2015,wang_mixing_2014,gaddam_tiling_2015}.
In these papers, the video is cut into independent tiles (typically $8\times 8$) and the server offers several
video qualities for each tile. The client selects the tile qualities according to the eye attention.

\subsection{Our Contributions}


%%% Local Variables:
%%% mode: latex
%%% TeX-master: "paper"
%%% End:
