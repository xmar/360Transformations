\section{Introduction}
\label{sec:introduction}

The popularity of navigable 360-degree video systems has grown with
the advent of omnidirectional capturing systems
and interactive displaying systems, like \acp{HMD}. However, to
deliver 360-degree video content on the Internet, the content
providers have to deal with a problem of bandwidth waste: What is
displayed on the device, which is indifferently called
\textit{\ac{FoV}} or \textit{viewport}, is only a fraction of what is
downloaded, which is an omnidirectional view of the scene.
This bandwidth waste is the price to pay for interactivity. To prevent
\emph{simulator sickness}~\cite{moss2011characteristics} and to
provide good \ac{QoE}, the vendors of \acp{HMD} recommend that the
enabling multimedia systems react to head movements as fast as the
\ac{HMD} refresh rate.
%\footnote{
%\url{https://developer.oculus.com/documentation/intro-vr/latest/concepts/bp_intro/}}
Since the refresh rate of state-of-the-art \acp{HMD} is
\SI{120}{\hertz},
%\footnote{\url{http://www.vrnerds.de/vr-brillen-vergleich/}}
the whole
system should react in less than \SI{10}{ms}. This delay constraint
prevents the implementation of traditional delivery architectures
where the client notifies a server about changes and awaits for the
reception of content adjusted at the server. Instead, in the current
\ac{VR} video delivery systems, the server sends the full $360$-degree
stream, from which the \ac{HMD} extracts the viewport in real time,
according to the user head movements. Therefore, the majority of the
delivered video stream data are not used.

Let us provide some numbers to illustrate this problem. The viewport
is defined by a device-specific viewing angle (typically
$120$ degrees), which delimits horizontally the scene from the head direction center, called \FoV{} center. To ensure a
good immersion, the pixel resolution of the displayed viewport is high,
typically $4$K ($3840\times2160$). So the
resolution of the full $360$-degree video is at least $12$K
($11520\times6480$). In addition, the immersion requires a video frame
rate on the order of the \ac{HMD} refresh rate, so typically around
\SI[mode=text]{100}{\acp{fps}}. Overall, high-quality $360$-degree
videos combine both a very large resolution (up to $12$K) and a very
high frame rate (up to \SI[mode=text]{100}{\acp{fps}}). To compare,
the bit-rate of 8K videos at \SI[mode=text]{60}{\acp{fps}} encoded
using \ac{HEVC} is around \SI{100}{Mbps}~\cite{7398367}.

We propose in this paper a solution where, following the same
principles as in rate-adaptive streaming technologies, the server
offers multiple \emph{representations} of the same $360$-degree video.
But instead of offering representations that only differ by their
bit-rate, the server offers here representations that differ by having
a better quality in a given region of the video. Our proposal is a
\emph{viewport-adaptive streaming system} and is depicted in
Figure~\ref{fig:deliverychain}. Each video representation is characterized
by a \emph{\ac{QEC}}, which is a given viewing position in the
spherical video. Around the \ac{QEC}, the quality of the video is
maximum, while it is lower for video parts that are far from the
\ac{QEC}. Similarly as in \ac{DASH}, the video is cut into segments
and the client periodically runs an \emph{adaptive algorithm} to
select a representation for the next segment. In a
viewport-adaptive system, clients select the representation
such that the bit-rate fits their receiving
bandwidth and the \ac{QEC} is closest to their \FoV{} center.

\begin{figure}
   \centering
   \tikzsetnextfilename{deliveryChain}
\begin{tikzpicture}

% this is for the guy looking at the right
\tikzset{
	rightVR/.pic = {
	% the only parameter is the size
		% first the body
		\draw plot [smooth] coordinates {(0,0)
			(0.05*#1,0.2*#1)
			(0.25*#1,0.2*#1)
			(0.3*#1,0)};
		\draw (0,0) -- (0.3*#1,0);
		% then the head
		\draw[fill=white] (0.15*#1, 0.4*#1) ellipse (0.15*#1 cm and 0.2*#1 cm);
		% then the VR headset
		\draw[rounded corners=0.3*#1, fill=black] (0.33*#1, 0.55*#1) rectangle
			(0.2*#1, 0.35*#1);
		\begin{scope}
			\clip (0.15*#1, 0.4*#1) ellipse (0.15*#1 cm and 0.2*#1 cm);
			\draw[fill=black] (0.2*#1, 0.5*#1) rectangle (0, 0.46*#1);
			\draw plot [smooth] coordinates {(0.22*#1,0.3*#1)
			 (0.27*#1,0.27*#1)
			 (0.32*#1,0.3*#1)};
		\end{scope}
	}
}

% this is for the guy looking at the front
\tikzset{
	frontVR/.pic = {
	% the only parameter is the size
		% first the body
		\draw plot [smooth] coordinates {(0,0)
			(0.08*#1,0.2*#1)
			(0.42*#1,0.2*#1)
			(0.5*#1,0)};
		\draw (0,0) -- (0.5*#1,0);
		% then the head
		\draw[fill=white] (0.25*#1, 0.4*#1) ellipse (0.15*#1 cm and 0.2*#1 cm);
		\draw plot [smooth] coordinates {(0.18*#1,0.3*#1)
			 (0.25*#1,0.27*#1)
			 (0.32*#1,0.3*#1)};
		% then the VR headset
		\draw[rounded corners=0.5*#1, fill=black] (0.07*#1, 0.55*#1) rectangle
			(0.43*#1, 0.35*#1);
	}
}

\tikzset{
     element/.style={
     	rounded corners,
     	rectangle,
  	 	thick,
  	 	draw=black,
  	 	minimum height=2cm,minimum width=2.5cm
     }
}

\tikzset{
	elementtitle/.style={
		rectangle,
		rounded corners,
		fill=gray!80,
		font=\footnotesize,
		text=white,
		anchor=north
	}
}

\tikzset{
	pics/equirec/.style n args={3}{
		code={
			\draw[fill=gray!30] (-0.0352778*#1, -0.019844*#1) rectangle (0.0352778*#1, 0.019844*#1);
			%\draw[draw=none,fill=gray!70] (0.0088194*#2*#1-2*0.0088194*#1, 0.0066147*#3*#1 - 2*0.0066147*#1) rectangle (0.0088194*#2*#1 + 2*0.0088194*#1, 0.0066147*#3*#1 + 2*0.0066147*#1);
			\draw[draw,fill=none] (-0.0352778*#1, -0.019844*#1) rectangle (0.0352778*#1, 0.019844*#1);
			%\draw[color=black,fill=black] (0.0088194*#2*#1, 0.0066147*#3*#1) circle (1pt);
		}
	}
}

\def\convCmPt{0.0352778}
\def\convCmPtRec{0.019844}
\def\convCmPtRecThird{0.0066147}
\def\convCmPtFourth{0.0088194}

\tikzset{cross/.style={cross out, draw,
         minimum size=2*(#1-\pgflinewidth),
         inner sep=0pt, outer sep=0pt}}

\tikzset{
	fov/.pic ={
		\draw[densely dotted, thick, red!70!black] (-0.07,0.10) rectangle (0.37,-0.20);
%		\draw[fill=red] (0.2,-0.05) circle (2pt);
		\draw (0.15,-0.05) node[cross=2pt,red!70!black] {};
	}
}


\tikzset{
	vr/.pic = {
		\draw[rounded corners] (-0.0352778*#1, -0.019844*#1) rectangle (0.0352778*#1, 0.019844*#1);
		\draw[rounded corners, thick] (-0.032*#1, -0.019844*#1) rectangle (0.032*#1, 0.016*#1);
%		\draw(-0.019*#1,0) pic {fov};
%		\draw(0.019*#1,0) pic {fov};
		\node[font=\scriptsize,rectangle,red, draw=red, thick,
					densely dotted, anchor=east, inner sep=2pt,
					yshift=-1pt, xshift=-1pt] at (0,0) {L};
		\node[font=\scriptsize,rectangle,red, draw=red, thick,
					densely dotted, anchor=west, inner sep=2pt,
					yshift=-1pt,xshift=0.5pt] at (0,0) {R};
	}
}

\def\ecartElement{16pt}
\def\sizeSphere{11}
\def\ecartObjet{2}
\def\ecartYVersions{16pt}


% capturing system
\node[element] (0,0) (capturing) {};
\node[elementtitle, above=-5pt of capturing] {capturing};
\draw ([xshift=-\sizeSphere - \ecartObjet pt]capturing.east) pic {spherical=\sizeSphere};
\pgfdeclareimage[width=18 pt]{camera}{plots/pictures/video-camera-icon-hi.png}
\node at ([xshift=21 pt]capturing.west) (camera1)
    {\pgfbox[right,center]{\pgfuseimage{camera}}};

% the arrow in the capture
\draw[-latex] (camera1.east) to ([xshift=-2*\sizeSphere - 2*\ecartObjet pt]capturing.east);

\onslide<.(1)-> {
% ===== server
\node[element,right=\ecartElement of capturing] (server) {};
\node[elementtitle, above=-5pt of server] {\vphantom{pt}server};
\draw ([xshift=\sizeSphere + \ecartObjet pt]server.west) pic {spherical=\sizeSphere};
\draw ([xshift=-\sizeSphere - \ecartObjet pt]server.east) pic {equirec={\sizeSphere}{2}{1}};
%\draw ([xshift=-\sizeSphere - \ecartObjet pt, yshift=\ecartYVersions]server.east) pic {equirec={\sizeSphere}{-2}{-1}};
%\draw ([xshift=-\sizeSphere - \ecartObjet pt, yshift=-\ecartYVersions]server.east) pic {equirec={\sizeSphere}{0}{0}};

% the three arrows in the server
\draw[-latex] ([xshift=2*\sizeSphere + 2*\ecartObjet pt]server.west) to ([xshift=-2*\sizeSphere - 2*\ecartObjet pt]server.east);
%\draw[-latex] ([xshift=2*\sizeSphere + 2*\ecartObjet pt]server.west) to ([xshift=-2*\sizeSphere - 2*\ecartObjet pt, yshift=\ecartYVersions]server.east);
%\draw[-latex] ([xshift=2*\sizeSphere + 2*\ecartObjet pt]server.west) to ([xshift=-2*\sizeSphere - 2*\ecartObjet pt, yshift=-\ecartYVersions]server.east);



% between capture and server
\draw[-latex] ([xshift=2pt]capturing.east) to ([xshift=-2pt]server.west);
}

\onslide<.(2)-> {
% ===== client
\node[element,right=\ecartElement of server] (client) {};
\node[elementtitle, above=-5pt of client] {\vphantom{pt}client};
% the equirecntagular
\draw([xshift=\sizeSphere + \ecartObjet pt]client.west) pic {equirec={\sizeSphere}{0}{0}};
% the old vr
%\draw ([xshift=-\sizeSphere - \ecartObjet pt]client.east) pic {vr=\sizeSphere};
% the fov
\onslide<.(2)>{
   \pic[local bounding box=thisfov] at ([xshift=\sizeSphere + \ecartObjet pt]client.west) {fov};
   \pic[local bounding box=headset] at ([xshift=-28 pt, yshift=-15pt]client.east) {rightVR=1.5};
}
\onslide<.(3)>{
   \pic[local bounding box=thisfov2] at ([xshift=\sizeSphere - \ecartObjet pt]client.west) {fov};
   \pic[local bounding box=headset] at ([xshift=-28 pt, yshift=-15pt]client.east) {frontVR=1.5};
}
%\node[anchor=west, font=\tiny,red, text width=23pt,align=center]
%		at ([yshift=10pt]client.west) (fovleg) {\ac{FoV} and \ac{FoV} center};
%
%\draw[red] (fovleg) -- (thisfov);

% vr headset



% old vr
%

% the arrow in the client
\onslide<.(2)>{
\draw[-latex]
%	([xshift=2*\sizeSphere + 2*\ecartObjet pt, yshift=-\ecartYVersions]client.west) to
	(thisfov.east) |- (headset);
		%([xshift=-2*\sizeSphere - 2*\ecartObjet pt]client.east);
      }
\onslide<.(3)>{
\draw[-latex]
%	([xshift=2*\sizeSphere + 2*\ecartObjet pt, yshift=-\ecartYVersions]client.west) to
	(thisfov2.east) |- (headset);
		%([xshift=-2*\sizeSphere - 2*\ecartObjet pt]client.east);
      }


% between server and client yshift=-\ecartYVersions
\draw[-latex] ([xshift=2pt]server.east) to ([xshift=-2pt]client.west);
}

\end{tikzpicture}

   \caption{Viewport-adaptive 360-degree video delivery system: The server
   offers video representations for three \acp{QEC}. The dark grey is the part of the video encoded at high quality, the light
   gray the low quality. The viewport is the dotted red rectangle, the \FoV{} center the
   cross}
   \label{fig:deliverychain}
\end{figure}

This viewport-adaptive $360$-degree streaming system has three
advantages: $(i)$ the bit-rate of the delivered video is lower than
the original full-quality video because video parts distant from the
\ac{QEC} are encoded at low quality. $(ii)$ When the end-user does not
move, the viewport is extracted from the highest quality part of the
spherical video. And $(iii)$ when the head of the end-user moves, the
device can still extract a viewport because it has the full
spherical video. If the new \FoV{} center is far from the \ac{QEC}
of the received video representation, the quality of the
extracted viewport is lower but this degradation holds only until the
selection of another representation with a closer \ac{QEC}.

The remainder of the paper is organized as follows. First, we
present our viewport-adaptive streaming
system, and
we show how it can be integrated into
the \ac{MPEG} \ac{DASH}-VR standard. Our proposal is thus a contribution
to the \ac{VR} group that \ac{MPEG} launched in May
$2016$~\cite{mpeg-vr}. Second, we address the choice of the geometric
layout into which the spherical video is projected for
encoding. We evaluate several video quality arrangements for a given
geometric layout and show that the cube map layout with full quality around the \ac{QEC} and \SI{25}{\percent} of this quality in the remaining faces offers the best quality of the extracted viewport.
Third, we study the required video segment length for
viewport-adaptive streaming. Based on a dataset of real users
navigating $360$-degree videos, we show that head movements occur over
short time periods, hence the streaming video segments have to be
short enough to enable frequent \ac{QEC} switches. Fourth, we
examine the impact of the number of \acp{QEC} on the viewport quality
and we show that a small number of (spatially-distributed over the sphere)
\acp{QEC} suffices to get high viewport quality.
Finally, we introduce a tool (released as open source), which creates video representations for the proposed viewport-adaptive streaming system.
The tool is highly configurable: from a given
$360$-degree video, it allows any arrangement of video quality for a
given geometric layout, and it extracts the viewport from any \FoV{} center.
This tool thus provides the main software module
for the implementation of viewport-adaptive streaming of navigable
$360$-degree videos.

%%% Local Variables:
%%% mode: latex
%%% TeX-master: "paper"
%%% End:
