\section{System Settings}
\label{sec:settings}

The preparation of the 360-degree video in our viewport-adaptive streaming system
requires several settings. We distinguish some global settings (the number of \acp{QEC} ($n$), 
the number of representations ($k$), the segment length and the geometric layout) 
and some settings \emph{per representation}
(the target bit-rate, the number of different qualities in the representation, the quality 
arrangement onto the different
faces). We do not aim
at being exhaustive in this paper. Some of these settings require a deeper study related to
signal processing and video encoding, while some other depend on business considerations and infrastructure investment. In this paper, we restrict our study
to three key questions: what is the best geometric layout to support quality-differentiated 360-degree
video? What is the best segment length to support head movements while maintaining low 
management overhead? What is the best number of \acp{QEC} $n$? To answer these three questions,
we have developed a software tool and we have used a dataset of real user watching 360-degree
video. 

%\subsection{Software and Dataset}

\parag{Dataset}We graciously got from Jaunt, Inc a dataset of the head movements
of real users watching a 360-degree video. The dataset is the same as
in~\cite{yu_framework_2015}. It consists of
ten omnidirectional videos of length ten seconds. These videos include
typical scenarios of 360-degree video. The dataset contains
the head movements of ten people who were asked to watch the videos on
a state-of-the-art \ac{HMD} (Occulus Rift DK2). The subjects were
standing and they were given the freedom to turn around, so the head
movements are of wider importance than if they were asked to watch the
video while sitting. Given the length of the video and the experiment
conditions, we believe that the head movements thus correspond to a
configuration of abrupt and wide head movements, which is the most
challenging case for our viewport-adaptive streaming system.
\citet{yu_framework_2015} have studied the most frequent head position of users.
In our case, we are interested in head movements during the
length of a segment.
%They show that the FoV center is around the equator of the spherical
%video during the majority of time but that lateral movements are
%frequent.

\parag{Software}We have developed our own tool to manipulate the main concepts
of viewport-adaptive streaming. Since the code is publicly available,%
\footnote{\url{https://github.com/xmar/360Transformations}} the software can be used and enhanced
to make further studies and to develop real systems. The main
features include: $(i)$ \emph{Projection from a spherical video into any of the four
geometric layouts and vice versa}. The spherical video is the pivot format from which it
is possible to project in any layout.
Note that the majority of 360-degree videos that are currently available are encoded and stored after
an equirectangular projection, but our tools enables re-projecting these videos
in another layout without information loss. $(ii)$ \emph{Adjusting video qualities for each 
geometric face of any
layout}. We set the resolution of the face in number of
pixels and the encoding bitrate goal. Each face is encoded into its own video
track to get different
bit-rate goal per face. And $(iii)$ \emph{Viewport extraction for any \ac{FoV} center in the
sphere}. It includes the management of video bit-rates and resolutions
when the extracted viewport overlaps several faces with different qualities.

\subsection{Geometric Layout}

We report now the experiment of measuring the video quality of viewports, which
are extracted from 360-degree videos projected onto various geometric layouts and 
with various
face quality arrangements. We evaluate video quality against two reference videos.
\begin{itemize}[leftmargin=7pt, itemindent=0pt, topsep=2pt, itemsep=0pt]	
\item \emph{The original video at full quality}. We used the
``\emph{Bridge
jumping}''\footnote{\url{https://youtu.be/yarcdW91djQ}} video. We can extract viewports at 1080p resolution from this 4k equirectangular
video. This original video 
is the reference for the objective video quality assessment tools: \ac{MS-SSIM} and
\ac{PSNR}. Since we compare several encoded versions of the same video against an original
one, these tools provide
a fair performance evaluation.
\item \emph{The same video re-encoded at a target bit-rate}. We set
a bit-rate target (here \SI{6}{\mega bps}). We 
re-encoded the original full-quality video with \ac{HEVC} by specifying this bit-rate target. We call it \emph{uniEqui} to state that, in this
video, the quality is uniform. This video is the basic delivered stream in a regular delivery
system.
%ith this bit-rate
%\emph{budget}, our competitor is the original equirectangular video re-encoded with 
%\ac{HEVC} by specifying this bit-rate target. We call it \emph{uniEqui} to state that, in this
%video, the quality is uniform. 
\end{itemize}

%Recall that our ultimate motivation 
%is to reduce the bit-rate of the delivered video, while maintaining a high video quality.

A \textit{quality-differentiated layout} is the combination of a geometric layout and video quality 
arrangement onto the geometric faces. The performance of the layout can be studied with
regards to two scales: \emph{the best viewport quality}, which is the quality of the extracted viewport 
when the FoV center and the QEC perfectly matches, and the \emph{sensitivity to head movements}, 
which is the degradation of the viewport quality when the distance between the FoV center 
and the QEC increases. 
To evaluate both scales, we chose one origin point \ac{QEC} in the spherical video. And then we iterated
over the \emph{orthodromic distance} $d$, which increases from 0 to $\pi$ (the furthest point). For each
value $d$, we randomly picked ten \ac{FoV} centers at the distance $d$ from the origin \ac{QEC}
and we extracted the viewport, which we compared to the same viewport extracted from the 
full-quality original video to get an objective video quality measure.

\begin{figure}
    \pgfplotscreateplotcyclelist{My color list}{%
    {gray,solid, very thick},%
    {black!25, dashed, very thick},%
    {black!75, densely dashed, very thick},%
    {black, densely dotted, very thick},%
}

\pgfplotsset{every axis legend/.append style={
        at={(0,0.97)},
        anchor=south west,
        draw=none,
        fill=none,
        legend columns=4,
        column sep=5pt,
        /tikz/every odd column/.append style={column sep=0cm},
        font=\scriptsize
}}
%\pgfplotsset{grid style={dashed,gray}}
\pgfplotsset{minor grid style={gray!20}}
\pgfplotsset{major grid style={dotted}}


\begin{tikzpicture}
    \begin{axis}[
            ylabel={PSNR gap (in dB)},
            xlabel={Orthodromic distance (in distance units)},
            width=1.05\linewidth,
            height=0.5\linewidth,
            cycle list name={My color list},
			 extra y ticks={0},
            extra y tick labels={\textit{\vphantom{lj}uniEqui}},
            extra y tick style={grid=major, 
               tick pos=left, 
               ticklabel pos=left, tick
               align=inside,
               grid=minor,
               tick label style={rounded corners,
                  fill=gray!20,
                  font=\tiny, 
                  inner sep=2pt,
                  anchor=west}},
            legend cell align=left,
            xmin = 0,
            xmax = 3.14,
%            ymajorgrids,
        ]

        \addplot+ table [x=distance,
        y=qualityEquirectangularTiledLower]{results/distanceQuality_psnr.csv};
        \addplot+ table [x=distance,
        y=qualityCubeMapLower]{results/distanceQuality_psnr.csv};
        \addplot+ table [x=distance,
        y=qualityPyramidLower]{results/distanceQuality_psnr.csv};
        \addplot+ table [x=distance,
        y=qualityRhombicDodeca]{results/distanceQuality_psnr.csv};
        \legend{Equirec,CubeMap,Pyramid,Dodeca}

    \end{axis}
\end{tikzpicture}

    \caption{Average \acs{PSNR} gap compared to the \emph{uniEqui} layout, depending on the distance from the \acs{QEC}. Global bitrate budget \SI{6}{\mega bps}}
    \label{fig:dist_quality_psnr}
\end{figure}

We represent in Figure~\ref{fig:dist_quality_psnr} the difference of \ac{PSNR} between
the extracted viewport from our quality-differentiated layouts and the same viewport extracted from
the reference \textit{uniEqui} layout. For each geometric layout (equirectangular 
panorama with $8\times 8$ tiles, cube map, pyramid, and dodecahedron), we have tested 
numerous quality arrangements with respect to the overall bit-rate budget. We selected
here the ``best" arrangement for each layout.

The projection on a cube map appears to be the best choice for the content provider. The quality of
the viewport when QEC and FoV center matches ($d=0$) is 3dB better than the viewport extracted
from the reference uniformly encoded 360-degree video at the same bit-rate. The gain in quality 
decreases when the distance $d$ increases but the quality of the cube map layout is always greater 
than the other layouts. Note that the pyramid projection (which is the layout chosen 
by Facebook~\cite{facebook}) is more sensitive to head movements than
the other layouts, due to the projection under-sampling. The video quality is worse than
the standard \emph{uniEqui} for distance greater than $1.2$. This choice requires thus
that the segment length are short enough to prevent head movements greater than this
distance during the length of a segment.

\subsection{Segment Length}

The segment length is a key
aspect of viewport-adaptive streaming. Long segment are easier to manage and better for video encoding, 
but short segments enable fast re-synchronisation to head movement. With respect to 
Figure~\ref{fig:dist_quality_psnr}, the 
segment length should be chosen so that the 
distance between the FoV center and the QEC are rarely greater than 1.5 distance units.

Based on the dataset, we show the distribution of head movements for various segment lengths
in Figure~\ref{cdf-dataset}. For each video and each person watching
it, we set timestamps that correspond to the starting time of a video
segment, \textit{i.e.,} the time at which the users select a QEC. Then, we measure the orthodromic distance
between this initial head position and every FoV center during the $x$
next seconds, where $x$ is the segment length. In
Figure~\ref{cdf-dataset}, we show the \ac{CDF} of the time spent at a
distance $d$ from the initial head position. A point at $(1.5,0.6)$
means that, on average, users spend $60\%$ of their time with a FoV
center at less than $1.5$ distance units from the FoV center on the
beginning of the segment.

\begin{figure}[htbp]
\centering
\pgfplotscreateplotcyclelist{My color list}{%
    {black!25,solid, very thick},%
    {black!50,densely dashed, very thick},%
    {black!75,densely dotted, very thick},%
    {black!100,dash pattern=on 4pt off 1pt on 4pt off 4pt, very thick}%
}

\pgfplotsset{every axis legend/.append style={
        at={(0.05,0.97)},
anchor=south west,
draw=none,
fill=none,
legend columns=4,
column sep=15pt,
/tikz/every odd column/.append style={column sep=0cm},
%font=\tiny
}}

\pgfplotsset{grid style={dashed,gray}}
\pgfplotsset{minor grid style={dotted,red!20!gray}}
\pgfplotsset{major grid style={dotted,gray!50!black}}

\tikzsetnextfilename{cdf_head_position_dataset}
\begin{tikzpicture}
    \begin{axis}[
            ylabel={CDF},
            xlabel={Orthodromic Distance to the Initial Head Position},
            width=0.95\linewidth,
            height=0.5\linewidth,
            %x=2cm,
    	    xtick={0,1, 2, 3},
            ytick={0,0.2,0.4,...,1},
            %x tick label as interval,
            %xticklabels={%
            %    {500~ms},%
            %    {1000~ms},%
            %    {1500~ms},%
            %    {2000~ms},%
            %},
            enlarge x limits=0.02,
            cycle list name={My color list},
            legend cell align=left,
            %xmin = 0,
            ymin = 0,
            %xmax = 3,
            ymax = 1,
            ymajorgrids,
            xmajorgrids,
            y filter/.code={\pgfmathparse{#1/100}\pgfmathresult},
        ]


%        \foreach \window in {0.1s, 0.25s, 0.5s, 1s, 1.5s, 2s, 2.5s,
%            3s, 3.5s, 4s, 4.5s, 5s} {
			\foreach \window in {1s, 2s, 3s, 5s}{
            \addplot+ table [x =\window, y = cdf]{results/global.csv};
        }
        	\legend{1\,s, 2\,s, 3\,s, 5\,s}
%        \legend{0.1s, 0.25s, 0.5s, 1s, 1.5s, 2s, 2.5s, 3s, 3.5s, 4s,
%        4.5s, 5s}

    \end{axis}
\end{tikzpicture}

\caption{CDF of the time spent at distance $d$ from the head position on the beginning of the
segment, for various segment lengths.}\label{cdf-dataset}
\end{figure}

Our main observation is that viewport-adaptive streaming requires
short segment lengths, typically smaller than 3\,s. Indeed, for a
segment length of five seconds, users spend on average half of their
time watching at a position that is at more than 1.3 distance units
away from the initial head position, which results in a degraded video
quality. A segment length of \SI{2}{\second} appears to be a good trade-off: 
85\% of
users never diverged to a head position that is further than 1.5~
distance unit away from the initial head position, and users can experience the full
video quality half of the time (head distance lesser than 0.5~distance unit). Please recall that
these experiments represent the most challenging (worst) case scenario
for our system. We can expect less abrupt and narrower head movements,
and thus longer possible segment lengths, for sitting users and longer
videos.

%such
%a length is acceptable regarding the management at the server (the
%number of segments to deal with and the size of the \ac{MPD}) and at
%the client (the frequency of representation selection and the number
%of requests to send). Furthermore, for a \SI{2}{\second}-long segment, 

\subsection{Number of \acp{QEC}}

The number of QECs ($n$) is another key trade-off. The more QECs, the better coverage
of the spherical video, and thus the better viewport quality due to a better match 
between QEC and FoV centers. However, the more QECs also means more storage
at the server side (to be multiplied by the number of bit-rate representations) and
more management (including longer \ac{MPD}).

We show in Figure~\ref{fig:QEC} the average \acs{MS-SSIM} quality of the viewports as
would the users of our datasets see them if the 360-degree video had been prepared
with a given number of QECs $n$. We used the best configuration, \textit{i.e.}, a cube map projection
and \SI{2}{\second} segment. We computed $n$ QEC positions in the video (uniform distribution over the sphere) and we generated $n$ quality-differentiated video representations from these $n$ QECs. 
For each head position in the dataset, we compute the distance between the
FoV center and the QEC that was chosen at the beginning of the segment and we compute
the viewport quality accordingly.

\begin{figure}[htbp]
\centering
\pgfplotscreateplotcyclelist{My color list}{%
    {black!25,solid, very thick},%
    {black!50,densely dashed, very thick},%
    {black!75,densely dotted, very thick},%
    {black!100,dash pattern=on 4pt off 1pt on 4pt off 4pt, very thick}%
}

\pgfplotsset{every axis legend/.append style={
        at={(0.05,0.97)},
anchor=south west,
draw=none,
fill=none,
legend columns=4,
column sep=15pt,
/tikz/every odd column/.append style={column sep=0cm},
%font=\tiny
}}

\pgfplotsset{grid style={dashed,gray}}
\pgfplotsset{minor grid style={dotted,red!20!gray}}
\pgfplotsset{major grid style={dotted,gray!50!black}}

\tikzsetnextfilename{cdf_head_position_dataset}
\begin{tikzpicture}
    \begin{axis}[
            ylabel={CDF},
            xlabel={Orthodromic Distance to the Initial Head Position},
            width=0.95\linewidth,
            height=0.5\linewidth,
            %x=2cm,
    	    xtick={0,1, 2, 3},
            ytick={0,0.2,0.4,...,1},
            %x tick label as interval,
            %xticklabels={%
            %    {500~ms},%
            %    {1000~ms},%
            %    {1500~ms},%
            %    {2000~ms},%
            %},
            enlarge x limits=0.02,
            cycle list name={My color list},
            legend cell align=left,
            %xmin = 0,
            ymin = 0,
            %xmax = 3,
            ymax = 1,
            ymajorgrids,
            xmajorgrids,
            y filter/.code={\pgfmathparse{#1/100}\pgfmathresult},
        ]


%        \foreach \window in {0.1s, 0.25s, 0.5s, 1s, 1.5s, 2s, 2.5s,
%            3s, 3.5s, 4s, 4.5s, 5s} {
			\foreach \window in {1s, 2s, 3s, 5s}{
            \addplot+ table [x =\window, y = cdf]{results/global.csv};
        }
        	\legend{1\,s, 2\,s, 3\,s, 5\,s}
%        \legend{0.1s, 0.25s, 0.5s, 1s, 1.5s, 2s, 2.5s, 3s, 3.5s, 4s,
%        4.5s, 5s}

    \end{axis}
\end{tikzpicture}

\caption{Average \acs{MS-SSIM} quality of the viewports of users depending 
on the number of QECs. Video projected on a cube map with \SI{2}{\second}-long segment. 
Bit-rate: \SI{6}{\mega bps}}
\label{fig:QEC}
\end{figure}

The best number of QECs in this configuration is between $xx$ and $yy$.