\section{Dataset and Head Movements}
\label{sec:dataset}

One of the critical aspects of our proposal is the sensibility to head movement. Once a QEC is chosen and the 
segment selected, any head movement that increases the distance between the QEC and the FoV center 
may result in a degradation of the video quality until the next segment selection time. 
Similarly as in rate-adaptive
streaming, the duration of the segment is a key setting of the system. The longer is the segment, the easier is
the management of the system (at both client and server sides), however the less reactive to changes is the system.
For FoV-adaptive streaming, we should deal with the same trade-off between management and adaptation 
as in rate-adaptive streaming,
except that the changes come from 
the navigation of the end-users (instead of network conditions).

To validate the principle behind FoV-adaptive streaming and to estimate adequate values for
the segment length, we have used a dataset of the head movements of real users watching
a 360-degree video. The dataset is the same as in~\cite{yu_framework_2015}. It comes from
Jaunt, Inc and consists of ten omnidirectional videos of length ten seconds. These videos include
different typical scenarios of 360-degree video. The dataset contains the head movements of
ten people who were asked to watch the videos on a state-of-the-art \ac{HMD} (Occulus Rift DK2).
The subjects were standing and they were given the freedom to turn around, so the head movements
are of wider importance than if they were asked to watch the video while sitting. Given the length of
the video and the experiment conditions, we believe that the head movements thus correspond to 
a configuration of abrupt and wide head movements, which is a worst case for our FoV-adaptive streaming.

\citet{yu_framework_2015} have studied the head position of users, in particular where does the FoV center lie 
during the video playback. They show that the FoV center is around the equator of the 360-degree video
during the majority of time but that lateral movements are frequent. In our case, we are interested in head
movements during the length of a segment.

We show the distribution of head movements for various segment lengths in Figure~\ref{cdf-dataset}. 
For each video
and each people watching it, we set timestamps that correspond to the starting time of a video segment,
\textit{i.e.} the time at which the users select a QEC based on their FoV centers.
Then, we measure the \emph{orthodromic distance} between this initial head position and every FoV center
during the $x$ next seconds, where $x$ is the segment length. In \ac{DASH}, typical segment length ranges
from $1$ to $10$\,s. In Figure~\ref{cdf-dataset}, we show the \ac{CDF} of the time spent at a distance
$d$ from the initial head position. A point at $(1.5,0.6)$ means that, in average, users spend $60\%$ of
their time with a FoV center at less than $1.5$ from the FoV center at the beginning of the segment.

\begin{figure}[htbp]
\centering
\pgfplotscreateplotcyclelist{My color list}{%
    {black!25,solid, very thick},%
    {black!50,densely dashed, very thick},%
    {black!75,densely dotted, very thick},%
    {black!100,dash pattern=on 4pt off 1pt on 4pt off 4pt, very thick}%
}

\pgfplotsset{every axis legend/.append style={
        at={(0.05,0.97)},
anchor=south west,
draw=none,
fill=none,
legend columns=4,
column sep=15pt,
/tikz/every odd column/.append style={column sep=0cm},
%font=\tiny
}}

\pgfplotsset{grid style={dashed,gray}}
\pgfplotsset{minor grid style={dotted,red!20!gray}}
\pgfplotsset{major grid style={dotted,gray!50!black}}

\tikzsetnextfilename{cdf_head_position_dataset}
\begin{tikzpicture}
    \begin{axis}[
            ylabel={CDF},
            xlabel={Orthodromic Distance to the Initial Head Position},
            width=0.95\linewidth,
            height=0.5\linewidth,
            %x=2cm,
    	    xtick={0,1, 2, 3},
            ytick={0,0.2,0.4,...,1},
            %x tick label as interval,
            %xticklabels={%
            %    {500~ms},%
            %    {1000~ms},%
            %    {1500~ms},%
            %    {2000~ms},%
            %},
            enlarge x limits=0.02,
            cycle list name={My color list},
            legend cell align=left,
            %xmin = 0,
            ymin = 0,
            %xmax = 3,
            ymax = 1,
            ymajorgrids,
            xmajorgrids,
            y filter/.code={\pgfmathparse{#1/100}\pgfmathresult},
        ]


%        \foreach \window in {0.1s, 0.25s, 0.5s, 1s, 1.5s, 2s, 2.5s,
%            3s, 3.5s, 4s, 4.5s, 5s} {
			\foreach \window in {1s, 2s, 3s, 5s}{
            \addplot+ table [x =\window, y = cdf]{results/global.csv};
        }
        	\legend{1\,s, 2\,s, 3\,s, 5\,s}
%        \legend{0.1s, 0.25s, 0.5s, 1s, 1.5s, 2s, 2.5s, 3s, 3.5s, 4s,
%        4.5s, 5s}

    \end{axis}
\end{tikzpicture}

\caption{CDF of the time spent at distance $d$ from the head position at the beginning of the 
segment, for various segment lengths.}\label{cdf-dataset}
\end{figure}

The main observation is that FoV-adaptive streaming requires short segment lengths, typically 
smaller than 3\,s. Indeed, in our FoV-adaptive streaming system, the full video quality is obtained
when the FoV center is close to the QEC. Here, we show that, for a segment length of five seconds,
users spend in average half of their time watching at a position that is at more than 
1.3 of the initial head position,
which corresponds to a wide head movement and, as we will see later, a degraded video quality.
Segment length of 2\,s appears like a good trade-off: such a length is acceptable regarding the
management at the server (the number of segments to deal with and the size of the manifest)
and at the client (the frequency of representation selection and the number of requests
to send). And, for 2\,s-long segment, people watch
at a position that is less than 1 from the initial head position during three quarters 
of the time. Please note that, as previously said, the configuration of the measures is one of
the worst cases for the system. In the case of sitting people and longer videos, we can expect less
wide and less abrupt head movements, which would turn into longer segment lengths.