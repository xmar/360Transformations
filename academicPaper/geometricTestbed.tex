\section{Geometric Layout Evaluation}

Our objective is to identify the layout that offers the best performance regarding the quality of 
the extracted \ac{FoV} video and the bit-rate of the delivered 360-degree video. We first introduce 
the tool that we release to generate and exploit quality-differentiated
360-degree videos on geometric layouts. Then, we describe the testbed that we set up to evaluate the
performance of the four main geometric layouts. Finally, we present our first results.

\subsection{Our Tool to Project 360-degree Video Into Quality-Differentiated Layouts}

The code and all the details are available at \textit{hidden url}. The main features of this tool include:
\begin{itemize}
\item Projection from a spherical video into any of the four geometric layouts and vice versa. Our tool is in 
particular able to re-project the 360-degree videos that are encoded and stored in the equirectangular layout,
which corresponds to a large fraction of current catalogs of 360-degree videos. 
\item Adjusting video qualities for each geometric face of any layout by setting \noteGS{to be written}
\item \ac{FoV} extraction for any spherical coordinate and any angle, in particular for quality-differentiated videos.
\end{itemize}

The software can thus be used by the scientific community to study geometric layouts, quality arrangement over the tiles, and \ac{FoV} extraction strategies.

\subsection{Testbed Description}

The testbed that we
set up is depicted in Figure~\ref{fig:testbed}. From the original spherical video, we considered the 
projection 
into the four geometric layouts (equirectangular, cube map, pyramid, dodecahedron). For 
each layout, we built
$x$ video versions characterized by different \acp{QEC}. Then we simulated
a client watching in a specific direction. We selected, for each layout, the version such that the \ac{QEC}
is the closest to the given direction. We extracted the \ac{FoV} from this version. Finally, we compared
the quality of the four different \ac{FoV}videos against the reference video generated
without quality loss from the spherical video.

\begin{figure}[t]
\begin{tikzpicture}

\def\ecartx{1}
\def\ecartybig{0.5}
\def\ecartysmall{0.3}
\def\heimax{8*\ecartysmall+3*\ecartybig}
\def\sizeIcon{12}
\def\unitdodeca{0.11}

\pic [local bounding box = equiFirst] at (\ecartx,0.5*\heimax) 
		{equirectangular={\sizeIcon}{-1}{0}};
\pic [local bounding box = equiSecond, 
		below= \ecartysmall of equiFirst] {equirectangular={\sizeIcon}{0}{-1}};
%\pic [local bounding box = equiThird, 
%		below=\ecartysmall of equiSecond] {equirectangular={\sizeIcon}{0}{0}};

\pic [local bounding box = cubeFirst, below=\ecartybig of equiSecond] {cubemap=\sizeIcon};
\pic [local bounding box = cubeSecond, 
		below=\ecartysmall of cubeFirst] {cubemap=\sizeIcon};
%\pic [local bounding box = cubeThird, 
%			below=\ecartysmall of cubeSecond] {cubemap=\sizeIcon};

\pic [local bounding box = pyraFirst, below=\ecartybig of cubeSecond] {pyramid=\sizeIcon};
\pic [local bounding box = pyraSecond, 
		below=\ecartysmall of pyraFirst] {pyramid=\sizeIcon};
%\pic [local bounding box = pyraThird, 
%			below=\ecartysmall of pyraSecond] {pyramid=\sizeIcon};

\pic [local bounding box = dodecaFirst, below=\ecartybig of pyraSecond, 
		xshift=-3pt] {dodecahedron=\unitdodeca};
\pic [local bounding box = dodecaSecond, below=\ecartysmall of dodecaFirst, xshift=-6pt,
		yshift=2pt] {dodecahedron=\unitdodeca};
%\pic [local bounding box = dodecaThird, 
%			below=\ecartysmall of dodecaSecond] {dodecahedron=\unitdodeca};

% ==== sphere
\pic [local bounding box=spher, yshift=10pt,
		right=-\ecartx cm of pyraFirst] at (0,0) {spherical=10};
		
% == links
\draw[-latex] (spher) to ([xshift=-3pt, yshift=5pt]equiSecond.west);
\draw[-latex] (spher) to ([xshift=-3pt, yshift=5pt]cubeSecond.west);
\draw[-latex] (spher) to ([xshift=-3pt, yshift=5pt]pyraSecond.west);
\draw[-latex] (spher) to ([xshift=-3pt, yshift=5pt]dodecaSecond.west);

% ==== selected versions
\pic [local bounding box = chosenEqui, 
		right= \ecartx of equiFirst] {equirectangular={\sizeIcon}{-1}{0}};
\pic [local bounding box = chosenCube, right=\ecartx of cubeFirst] {cubemap=\sizeIcon};
\pic [local bounding box = chosenPyra, 
		right=\ecartx of pyraSecond] {pyramid=\sizeIcon};
\pic [local bounding box = chosenDodeca, right=\ecartx of dodecaFirst, 
		xshift=-5pt, yshift=3.5pt] {dodecahedron=\unitdodeca};
		
% == links
\draw[-latex] ([xshift=1.5pt]equiFirst.east) to ([xshift=-1pt]chosenEqui.west);
\draw[-latex] ([xshift=1.5pt]cubeFirst.east) to ([xshift=-1pt]chosenCube.west);
\draw[-latex] ([xshift=1.5pt]pyraSecond.east) to ([xshift=-1pt]chosenPyra.west);
\draw[-latex] ([xshift=0.5pt]dodecaFirst.east) to ([xshift=-1pt]chosenDodeca.west);

% == legend
\tikzset{
	legendNode/.style={
		font=\scriptsize,
		text width=1cm,
		align=center,
		anchor=south
	}
}
\node[legendNode] at (chosenEqui |- equiFirst.north) {selected version};
\node[legendNode] at (equiFirst.north) {offered versions};

% ==== fov

\tikzset{
	pics/equirec/.style n args={4}{
		code={
			\draw[draw=none,fill=gray!30] (-0.0352778*#1, -0.019844*#1) rectangle (0.0352778*#1, 0.019844*#1);
			\draw[draw=none,fill=gray!70] (0.0088194*#2*#1-#4*0.0088194*#1, 0.0066147*#3*#1 - #4*0.0066147*#1) rectangle (0.0088194*#2*#1 + #4*0.0088194*#1, 0.0066147*#3*#1 + #4*0.0066147*#1);
			\draw[draw, very thick, densely dotted,fill=none] (-0.0352778*#1, -0.019844*#1) rectangle (0.0352778*#1, 0.019844*#1);
		}		
	}
}

\pic [local bounding box=fovEqui, right=\ecartx of chosenEqui] {equirec={\sizeIcon}{-2}{-1}{2}};
\pic [local bounding box=fovCube, right=\ecartx of chosenCube] {equirec={\sizeIcon}{1}{0}{3}};
\pic [local bounding box=fovPyra, right=\ecartx of chosenPyra] {equirec={\sizeIcon}{0}{-1}{2}};
\pic [local bounding box=fovDodeca] at (fovPyra |- chosenDodeca) {equirec={\sizeIcon}{0}{0}{3}};

\node[legendNode] at (fovEqui |- equiFirst.north) {\ac{FoV} extraction};

% == links

\draw[-latex] ([xshift=1.5pt]chosenEqui.east) to ([xshift=-1pt]fovEqui.west);
\draw[-latex] ([xshift=1.5pt]chosenCube.east) to ([xshift=-1pt]fovCube.west);
\draw[-latex] ([xshift=1.5pt]chosenPyra.east) to ([xshift=-1pt]fovPyra.west);
\draw[-latex] ([xshift=0.5pt]chosenDodeca.east) to ([xshift=-1pt]fovDodeca.west);

% ==== reference video
\tikzset{
	refVideo/.pic={
		\draw[draw=black, very thick, densely dotted, fill=gray!70] (-0.0352778*#1, -0.019844*#1) rectangle (0.0352778*#1, 0.019844*#1);
	}
}

\pic[local bounding box=fovRef,yshift=-15pt] 
		at (fovDodeca |- dodecaSecond.south) {refVideo=\sizeIcon};
\node[legendNode, text width=2cm] at (fovRef.north) {reference video};
		
\draw[-latex] (spher.south) |- (fovRef.west);

% ==== comparison MS-SSIM

\node[rounded corners,rectangle,thick,fill=gray!10,
		text width=1.5cm,align=center,anchor=west] (msssim)
		at ([xshift=\ecartx cm]fovEqui |- spher) {MS-SSIM comparison};
		
% == links

\draw[-latex] (fovEqui.east) -| (msssim.45);
\draw[-latex] (fovCube.east) -| (msssim.135);
\draw[-latex] (fovPyra.east) -| (msssim.225);
\draw[-latex] (fovDodeca.east) -| (msssim.270);
\draw[-latex] (fovRef.east) -| (msssim.315);

\end{tikzpicture}

\caption{Our testbed: the spherical video is projected into four geometric layouts with variable qualities and
one layout at full-quality for reference. The best version is selected, and the \ac{FoV} is extracted. The quality
of the extracted \acp{FoV} are compared with respect to the reference video.}\label{fig:testbed}
\end{figure}

To generate the different videos, we made some choices, which are mostly in conformance to the
literature and to real implementation of 360-degree video delivery systems. Our future work 
includes to study in more details the impact of some of these choices on
the overall performance of the system; it is not our objective here.

We generated $x=16$ different video versions for each layout. This number of versions is smaller than
the number of videos (with pyramid layouts) given by~\citet{facebook}. It is however closer to the
number of video representations that are recommended in rate-adaptive streaming 
systems~\cite{Aparicio-PardoP15}. We describe now specific quality arrangements per layout:
%\begin{description}
%\item[equirectangular] 

\parag{Equirectangular}We cut the equirectangular layout into $8\times 8$ tiles as proposed in
the literature related to panorama video~\cite{gaddam_tiling_2015}. Tiles are thus rectangular.
We considered three different qualities: the \emph{full quality}, which corresponds to the quality of the
input spherical video, a \emph{medium quality}, which is set as half as good as the full quality, and a 
\emph{low quality}, which is one quarter of the full quality. The quality arrangement is done such that
the $3\times 3$ tiles that are around the \ac{QEC} are full quality, the $7\times 7$ tiles around 
the \ac{QEC} (excluding the closest $3\times 3$) are with the medium quality, and the remaining tiles
are with the low quality.

\parag{Cube Map}To generate the $x=16$ \acp{QEC}, we rotate the cube in the sphere so that the
center of the square front face matches the \ac{QEC}. The front face is at full quality, while the left, right, 
top, and bottom faces are mid quality and the back face is low quality.

\parag{Pyramid}This layout differs from other layouts. First, it does not preserve
the pixel information of the spherical videos. Second, the distortion depends on the size of the square
base face and on the distance from the peak. As for the cube map, we rotate the pyramid to adjust
the center of the base face to the \ac{QEC}. The base is at full quality while the other faces are
at a medium quality. \noteGS{To be written}

\parag{Rhombic Dodecahedron}We rotated the dodecahedron such that the \ac{QEC} is between
the two same faces (say face\,1 and face\,2). Both faces are at full quality. Then, the eight 
faces that are around
these faces are at medium quality, and the two remaining faces are at low quality.

We observed that, for the quality setting that we decided for the equirectangular (medium quality and 
low quality at half and a quarter quality of the full quality respectively), the bit-rate of the
generated equirectangular layouts are larger than for the other layouts. To get the
same bit-rate for each video version, we apply the following process. We set that, for a given 
360-degree video, the projection on the equirectangular provides the \emph{bit-rate budget}. For the
other layout, the low quality is kept at a quarter of the full quality, but we select the video quality of the medium quality such that the bit-rate of the generated video version is equal to the 
bit-rate budget.

Other settings are as follows. 
The 360-degree video that we chose is \noteGS{To be given}. The encoding of the video in each
geometric layout is done by HEVC \noteGS{To be given}. The direction of the client heads is as follows.
First we chose a \ac{FoV} center by a uniform random choice. Then we selected a second \ac{FoV} center
by a random choice that increases the chances to select a position near the equator. We consider
both points as the origin and the destination \ac{FoV} centers. The selection of the selected version
is based on the distance between the origin \ac{FoV} center and the \ac{QEC}. To compare the 
quality of the \ac{FoV} video against the reference video, we use the \emph{MS-SSIM} quality
evaluation tool.
