\section{Geometric Layout Evaluation}

Our objective is to identify the layout that offers the best
performance regarding the quality of the extracted \ac{FoV} video and
the bit-rate of the delivered 360-degree video. We first introduce the
tool that we release to generate and exploit quality-differentiated
360-degree videos on geometric layouts. Then, we describe the testbed
that we set up to evaluate the performance of the four main geometric
layouts. Finally, we present our first results.

\subsection{Our Tool to Project 360-degree Video Into
Quality-Differentiated Layouts}

The code and all the details are available at
\textit{https://github.com/xmar/360Transformations}. The main features
of this tool include:
\begin{itemize}
   \item Projection from a spherical video into any of the four
   geometric layouts and vice versa. Our tool is in particular able to
   re-project the 360-degree videos that are encoded and stored in the
   equirectangular layout, which corresponds to a large fraction of
   current catalogs of 360-degree videos.

   \item Adjusting video qualities for each geometric face of any
   layout by setting \XC{the resolution of the face in number of
   pixels and the encoding bitrate goal. In order to allow different
   bitrate goal per face, each face is encoded into its own video
   track. The equirectangular layout can be split into $8\times8$
   tiles that can each have their own resolution and encoding test
   bitrate.}{}

   \item \ac{FoV} extraction for any \XC{head positions}{}, in
   particular for quality-differentiated videos.
\end{itemize}

The software can thus be used by the scientific community to study
geometric layouts, quality arrangement over the tiles, and \ac{FoV}
extraction strategies.

\subsection{Testbed Description}

The testbed that we set up is depicted in Figure~\ref{fig:testbed}.
From the original spherical video, we considered the projection into
the four geometric layouts (equirectangular, cube map, pyramid,
dodecahedron). For each layout, we built $x$ video versions
characterized by different \acp{QEC}. Then we simulated a client
watching in a specific direction. \XC{For each video version of each
layout we extract the \ac{FoV} associated with the client head
position. We compute the orthodromic distance between the \ac{FoV}
center and the \ac{QEC} of the video version. We also extract the
\ac{FoV} from the original equirectangular video and from the
equirectangular video that have the same overall bitrate budget as the
other layouts. Finally we compared the quality of all those extracted
\ac{FoV} with the \ac{FoV} extracted from the original equirectangular
video.}

\begin{figure}
   \centering
   \begin{figure}[t]
\begin{tikzpicture}

\def\ecartx{1}
\def\ecartybig{0.5}
\def\ecartysmall{0.3}
\def\heimax{8*\ecartysmall+3*\ecartybig}
\def\sizeIcon{12}
\def\unitdodeca{0.11}

\pic [local bounding box = equiFirst] at (\ecartx,0.5*\heimax) 
		{equirectangular={\sizeIcon}{-1}{0}};
\pic [local bounding box = equiSecond, 
		below= \ecartysmall of equiFirst] {equirectangular={\sizeIcon}{0}{-1}};
%\pic [local bounding box = equiThird, 
%		below=\ecartysmall of equiSecond] {equirectangular={\sizeIcon}{0}{0}};

\pic [local bounding box = cubeFirst, below=\ecartybig of equiSecond] {cubemap=\sizeIcon};
\pic [local bounding box = cubeSecond, 
		below=\ecartysmall of cubeFirst] {cubemap=\sizeIcon};
%\pic [local bounding box = cubeThird, 
%			below=\ecartysmall of cubeSecond] {cubemap=\sizeIcon};

\pic [local bounding box = pyraFirst, below=\ecartybig of cubeSecond] {pyramid=\sizeIcon};
\pic [local bounding box = pyraSecond, 
		below=\ecartysmall of pyraFirst] {pyramid=\sizeIcon};
%\pic [local bounding box = pyraThird, 
%			below=\ecartysmall of pyraSecond] {pyramid=\sizeIcon};

\pic [local bounding box = dodecaFirst, below=\ecartybig of pyraSecond, 
		xshift=-3pt] {dodecahedron=\unitdodeca};
\pic [local bounding box = dodecaSecond, below=\ecartysmall of dodecaFirst, xshift=-6pt,
		yshift=2pt] {dodecahedron=\unitdodeca};
%\pic [local bounding box = dodecaThird, 
%			below=\ecartysmall of dodecaSecond] {dodecahedron=\unitdodeca};

% ==== sphere
\pic [local bounding box=spher, yshift=10pt,
		right=-\ecartx cm of pyraFirst] at (0,0) {spherical=10};
		
% == links
\draw[-latex] (spher) to ([xshift=-3pt, yshift=5pt]equiSecond.west);
\draw[-latex] (spher) to ([xshift=-3pt, yshift=5pt]cubeSecond.west);
\draw[-latex] (spher) to ([xshift=-3pt, yshift=5pt]pyraSecond.west);
\draw[-latex] (spher) to ([xshift=-3pt, yshift=5pt]dodecaSecond.west);

% ==== selected versions
\pic [local bounding box = chosenEqui, 
		right= \ecartx of equiFirst] {equirectangular={\sizeIcon}{-1}{0}};
\pic [local bounding box = chosenCube, right=\ecartx of cubeFirst] {cubemap=\sizeIcon};
\pic [local bounding box = chosenPyra, 
		right=\ecartx of pyraSecond] {pyramid=\sizeIcon};
\pic [local bounding box = chosenDodeca, right=\ecartx of dodecaFirst, 
		xshift=-5pt, yshift=3.5pt] {dodecahedron=\unitdodeca};
		
% == links
\draw[-latex] ([xshift=1.5pt]equiFirst.east) to ([xshift=-1pt]chosenEqui.west);
\draw[-latex] ([xshift=1.5pt]cubeFirst.east) to ([xshift=-1pt]chosenCube.west);
\draw[-latex] ([xshift=1.5pt]pyraSecond.east) to ([xshift=-1pt]chosenPyra.west);
\draw[-latex] ([xshift=0.5pt]dodecaFirst.east) to ([xshift=-1pt]chosenDodeca.west);

% == legend
\tikzset{
	legendNode/.style={
		font=\scriptsize,
		text width=1cm,
		align=center,
		anchor=south
	}
}
\node[legendNode] at (chosenEqui |- equiFirst.north) {selected version};
\node[legendNode] at (equiFirst.north) {offered versions};

% ==== fov

\tikzset{
	pics/equirec/.style n args={4}{
		code={
			\draw[draw=none,fill=gray!30] (-0.0352778*#1, -0.019844*#1) rectangle (0.0352778*#1, 0.019844*#1);
			\draw[draw=none,fill=gray!70] (0.0088194*#2*#1-#4*0.0088194*#1, 0.0066147*#3*#1 - #4*0.0066147*#1) rectangle (0.0088194*#2*#1 + #4*0.0088194*#1, 0.0066147*#3*#1 + #4*0.0066147*#1);
			\draw[draw, very thick, densely dotted,fill=none] (-0.0352778*#1, -0.019844*#1) rectangle (0.0352778*#1, 0.019844*#1);
		}		
	}
}

\pic [local bounding box=fovEqui, right=\ecartx of chosenEqui] {equirec={\sizeIcon}{-2}{-1}{2}};
\pic [local bounding box=fovCube, right=\ecartx of chosenCube] {equirec={\sizeIcon}{1}{0}{3}};
\pic [local bounding box=fovPyra, right=\ecartx of chosenPyra] {equirec={\sizeIcon}{0}{-1}{2}};
\pic [local bounding box=fovDodeca] at (fovPyra |- chosenDodeca) {equirec={\sizeIcon}{0}{0}{3}};

\node[legendNode] at (fovEqui |- equiFirst.north) {\ac{FoV} extraction};

% == links

\draw[-latex] ([xshift=1.5pt]chosenEqui.east) to ([xshift=-1pt]fovEqui.west);
\draw[-latex] ([xshift=1.5pt]chosenCube.east) to ([xshift=-1pt]fovCube.west);
\draw[-latex] ([xshift=1.5pt]chosenPyra.east) to ([xshift=-1pt]fovPyra.west);
\draw[-latex] ([xshift=0.5pt]chosenDodeca.east) to ([xshift=-1pt]fovDodeca.west);

% ==== reference video
\tikzset{
	refVideo/.pic={
		\draw[draw=black, very thick, densely dotted, fill=gray!70] (-0.0352778*#1, -0.019844*#1) rectangle (0.0352778*#1, 0.019844*#1);
	}
}

\pic[local bounding box=fovRef,yshift=-15pt] 
		at (fovDodeca |- dodecaSecond.south) {refVideo=\sizeIcon};
\node[legendNode, text width=2cm] at (fovRef.north) {reference video};
		
\draw[-latex] (spher.south) |- (fovRef.west);

% ==== comparison MS-SSIM

\node[rounded corners,rectangle,thick,fill=gray!10,
		text width=1.5cm,align=center,anchor=west] (msssim)
		at ([xshift=\ecartx cm]fovEqui |- spher) {MS-SSIM comparison};
		
% == links

\draw[-latex] (fovEqui.east) -| (msssim.45);
\draw[-latex] (fovCube.east) -| (msssim.135);
\draw[-latex] (fovPyra.east) -| (msssim.225);
\draw[-latex] (fovDodeca.east) -| (msssim.270);
\draw[-latex] (fovRef.east) -| (msssim.315);

\end{tikzpicture}

\caption{Our testbed: the spherical video is projected into four geometric layouts with variable qualities and
one layout at full-quality for reference. The best version is selected, and the \ac{FoV} is extracted. The quality
of the extracted \acp{FoV} are compared with respect to the reference video.}\label{fig:testbed}
\end{figure}
   \caption{Our testbed: the spherical video is projected into four geometric layouts with variable qualities and
   one layout at full-quality for reference. The best version is selected, and the \ac{FoV} is extracted. The quality
   of the extracted \acp{FoV} are compared with respect to the reference video.}
   \label{fig:testbed}
\end{figure}

To generate the different videos, we made some choices, which are
mostly in conformance to the literature and to real implementation of
$360$-degree video delivery systems. Our future work includes to study
in more details the impact of some of these choices on the overall
performance of the system; it is not our objective here.

\XC{ Each layout is made of faces: portions of the spherical for the
equirectangular layout with tiles and faces of the $3$D object for the
cube map, pyramid, and rhombic dodecahedron layouts. For each layout
we define a central face and we rotate the $3$D sphere so that the
\ac{QEC} of the representation is always located at the center of the
center of the central face. We fix a \emph{global bitrate budget}.
Then  each face is encoded with its own bitrate goal, fixed by the
definition of the layout, so that the global bitrate of the video fit
the global budget. It is not possible with current video encoder to
fix different bitrate goal to different spatial area of the same
video. To bypass this problem, each face is encoded into its own video
track and then all tracks for the same representation are stored into
the same video container. }

% \noteXC{This number of QEC is not right anymore. In the results we
% present, the number of QEC is not  important.} We generated $x=16$
% different video versions for each layout. This number of versions is
% smaller than the number of videos (with pyramid layouts) given
% by~\citet{facebook}. It is however closer to the number of video
% representations that are recommended in rate-adaptive streaming
% systems~\cite{Aparicio-PardoP15}. We describe now specific quality
% arrangements per layout:

\parag{Equirectangular with tiles}We cut the equirectangular layout into
$8\times8$ tiles as proposed in the literature related to panorama
video~\cite{gaddam_tiling_2015}. Tiles are thus rectangular. We
considered two different qualities: the \emph{full quality}, which
corresponds to the \XC{bitrate}{} of the input spherical video, and
a \emph{low quality}, which is one quarter of the full quality. The
quality arrangement is done such that the $3\times3$
tiles that are around the \ac{QEC} are full quality, and the
remaining tiles are with the low quality.

\parag{Cube Map} The central face is set to be the front face. The
front face is at full quality, while the left, right, top, bottom, and
back faces are low quality.

\parag{Pyramid} The pyramid layout is made of five faces: The square
base face and four triangle faces. The central face is the base face.
This layout differs from other layouts. First, it does not preserve
the pixel information of the original spherical videos. Information
are lost on the triangle faces. Second, the distortion depends on the
size of the square base face and on the distance from the peak.  The
base is at full quality while the other faces are at a low quality.
For our simulation we chose a base face of size $2.5$ distance units.

\parag{Rhombic Dodecahedron} The center of the layout is set on one of
the node of the Rhombic Dodecahedron. The three faces that contain
this node are at full quality and the other faces are at low quality.

\parag{Default equirectangular} To fairly compare our results with the
state-of-the-art, we encode the regular equirectangular video with the
same bitrate budget as the other layout. This layout will be our point
of reference. Please, note that the objective \ac{QoE} metric is
always computed using the original equirectangular video for
reference, not this re-encoded layout.

% We observed that, for the quality setting that we decided for the
% equirectangular (medium quality and low quality at half and a quarter
% quality of the full quality respectively), the bit-rate of the
% generated equirectangular layouts are larger than for the other
% layouts. To get the same bit-rate for each video version, we apply the
% following process. We set that, for a given 360-degree video, the
% projection on the equirectangular provides the \emph{bit-rate budget}.
% For the other layout, the low quality is kept at a quarter of the full
% quality, but we select the video quality of the medium quality such
% that the bit-rate of the generated video version is equal to the
% bit-rate budget.

% Other settings are as follows. The 360-degree video that we chose is
% \noteGS{To be given}. The encoding of the video in each geometric
% layout is done by HEVC \noteGS{To be given}. The direction of the
% client heads is as follows. First we chose a \ac{FoV} center by a
% uniform random choice. Then we selected a second \ac{FoV} center by a
% random choice that increases the chances to select a position near the
% equator. We consider both points as the origin and the destination
% \ac{FoV} centers. The selection of the selected version is based on
% the distance between the origin \ac{FoV} center and the \ac{QEC}. To
% compare the quality of the \ac{FoV} video against the reference video,
% we use the \emph{MS-SSIM} quality evaluation tool.


\subsection{Results}
\label{subsec:results}

To run this simulation we chose the following settings. We used the
``\textit{$4$K $360\degree{}$ Bridge
jumping}''\footnote{\url{https://youtu.be/yarcdW91djQ}} video from
Youtube. This video is a $4$K equirectangular video. From this video
we can only generate $1080$\acp{p} \acp{FoV}. Layout faces are encoded
with the \ac{HEVC} codec, using the \textit{x265} encoder with the
default parameters and with a fixed Average Bit-Rate goal.

\begin{figure}
    \pgfplotscreateplotcyclelist{My color list}{%
    {gray,solid, very thick},%
    {black!25, dashed, very thick},%
    {black!75, densely dashed, very thick},%
    {black, densely dotted, very thick},%
}

\pgfplotsset{every axis legend/.append style={
        at={(0,0.97)},
anchor=south west,
draw=none,
fill=none,
legend columns=-1,
column sep=15pt,
/tikz/every odd column/.append style={column sep=0cm},
%font=\tiny
}}
\pgfplotsset{grid style={dashed,gray}}
\pgfplotsset{minor grid style={dotted,red!20!gray}}
\pgfplotsset{major grid style={dotted,green!50!black}}


\begin{tikzpicture}
    \begin{axis}[
            ylabel={MS-SSIM},
            xlabel={Orthodromic distance},
            width=1.05\linewidth,
            height=0.5\linewidth,
 %           unit vector ratio=1 2 1,
 %           x=0.4cm,
            cycle list name={My color list},
            %ytick={0,2,4,...,10},
            %xtick pos=left,
            %x tick label style={at={(current axis.right of origin)},rotate=45,anchor=north east},
            %minor y tick num={1},
            legend cell align=left,
            xmin = 0,
            xmax = 3.14,
            %ymin = 0.7,
            ymax = 1,
            ymajorgrids,
            yminorgrids,
%            axis on top,
            %y filter/.code={\pgfmathparse{#1/1000000}\pgfmathresult},
        ]

        \addplot+ table [x=distance, y=qualityEquirectangularTiled]{results/distanceQuality.csv};
        \addplot+ table [x=distance, y=qualityCubMap]{results/distanceQuality.csv};
        \addplot+ table [x=distance, y=qualityPyramidal]{results/distanceQuality.csv};
        \addplot+ table [x=distance, y=qualityRhombicDodeca]{results/distanceQuality.csv};
        \addplot+ table [x=distance, y=qualityAverageEquiTi]{results/distanceQuality.csv};
        \legend{EqTiled,CubeMap,Pyramid,RhombicDodeca,AverageEqui}

    \end{axis}
\end{tikzpicture}

    \caption{Average \acs{MS-SSIM} depending on the distance from the
    \acs{QEC}. Global bitrate budget \SI{6}{\mega bps}, distance step
    $0.25$, test per point $4$, \SI{1}{\second} segment from the ``Bidge jumping'' video}
    \label{fig:dist_quality}
\end{figure}

\begin{figure}
    \pgfplotscreateplotcyclelist{My color list}{%
    {gray,solid, very thick},%
    {black!25, dashed, very thick},%
    {black!75, densely dashed, very thick},%
    {black, densely dotted, very thick},%
}

\pgfplotsset{every axis legend/.append style={
        at={(0,0.97)},
        anchor=south west,
        draw=none,
        fill=none,
        legend columns=4,
        column sep=5pt,
        /tikz/every odd column/.append style={column sep=0cm},
        font=\scriptsize
}}
%\pgfplotsset{grid style={dashed,gray}}
\pgfplotsset{minor grid style={gray!20}}
\pgfplotsset{major grid style={dotted}}


\begin{tikzpicture}
    \begin{axis}[
            ylabel={PSNR gap (in dB)},
            xlabel={Orthodromic distance (in distance units)},
            width=1.05\linewidth,
            height=0.5\linewidth,
            cycle list name={My color list},
			 extra y ticks={0},
            extra y tick labels={\textit{\vphantom{lj}uniEqui}},
            extra y tick style={grid=major, 
               tick pos=left, 
               ticklabel pos=left, tick
               align=inside,
               grid=minor,
               tick label style={rounded corners,
                  fill=gray!20,
                  font=\tiny, 
                  inner sep=2pt,
                  anchor=west}},
            legend cell align=left,
            xmin = 0,
            xmax = 3.14,
%            ymajorgrids,
        ]

        \addplot+ table [x=distance,
        y=qualityEquirectangularTiledLower]{results/distanceQuality_psnr.csv};
        \addplot+ table [x=distance,
        y=qualityCubeMapLower]{results/distanceQuality_psnr.csv};
        \addplot+ table [x=distance,
        y=qualityPyramidLower]{results/distanceQuality_psnr.csv};
        \addplot+ table [x=distance,
        y=qualityRhombicDodeca]{results/distanceQuality_psnr.csv};
        \legend{Equirec,CubeMap,Pyramid,Dodeca}

    \end{axis}
\end{tikzpicture}

    \caption{Average \acs{PSNR} gap compared to the Equirectagular layout encoded with the same bitrate budget depending on the distance from the \acs{QEC}. Global bitrate budget \SI{6}{\mega bps}, distance step $0.25$, test per point $4$, \SI{1}{\second} segment from the ``Bidge jumping'' video}
    \label{fig:dist_quality_psnr}
\end{figure}

In our first simulation we want to observe the \ac{QoE} degradation
depending on the distance between a \ac{FoV} center and a
\ac{QEC}. We consider the video is projected on the unit sphere. The
distance is measured using the orthodromic distance on this sphere. We
used a distance step of 0.25 units. For each distance $d$, for each
layout, and for each \ac{QEC} we randomly select $10$ \ac{FoV} center
located at the distance $d$ from the \ac{QEC} and we extract the
associated \acp{FoV}. Then we use the \ac{FoV} at the same position
extracted from the original video to compute the \ac{MS-SSIM} and the
\ac{PSNR} of each \ac{FoV}.

The Figure~\ref{fig:dist_quality} displays the average \ac{MS-SSIM} of
the \acp{FoV} depending on their distance from the different \acp{QEC}
and Figure~\ref{fig:dist_quality_psnr} represents the average
\ac{PSNR} gap from the equirectagular layout encoded with the same
bitrate budget. Both Figures are from the same simulation results. We
observe that the cube map layout generates the best results. The
\ac{QoE} from the pyramid layout decreases quickly: After a distance of
$1$ the quality is under the default equirectangular layout quality.
The quality of the cube map layout is never under the
quality of the default equirectangular layout. This was not expected. It may
be because we used tiles inside the equirectangular layout, this avoid
the encoder to efficiently compress the video. \noteXC{There is still
some simulation running to see if we can show better results}

\begin{figure}
    \pgfplotscreateplotcyclelist{My color list}{%
    {black!25,solid, very thick},%
    {black!50,densely dashed, very thick},%
    {black!75,densely dotted, very thick},%
    {black!100,dash pattern=on 4pt off 1pt on 4pt off 4pt, very thick},%
    {black!25,dotted, very thick},%
}

\pgfplotsset{every axis legend/.append style={
        at={(0,0.97)},
anchor=south west,
draw=none,
fill=none,
legend columns=2,
column sep=15pt,
/tikz/every odd column/.append style={column sep=0cm},
%font=\tiny
}}

\pgfplotsset{grid style={dashed,gray}}
\pgfplotsset{minor grid style={dotted,red!20!gray}}
\pgfplotsset{major grid style={dotted,gray!50!black}}

\tikzsetnextfilename{instant_distance}
\begin{tikzpicture}
    \begin{axis}[
            ylabel={QoE},
            xlabel={time (s)},
            width=1.05\linewidth,
            height=0.5\linewidth,
            %x=2cm,
%    	    xtick={-10,-5,0,5,...,50},
            %ytick={0,0.1,0.2,...,1},
            %x tick label as interval,
            %xticklabels={%
            %    {500~ms},%
            %    {1000~ms},%
            %    {1500~ms},%
            %    {2000~ms},%
            %},
            cycle list name={My color list},
            legend cell align=left,
            xmin = 0,
            %ymin = 0,
            xmax = 10,
            %ymax = 1,
            ymajorgrids,
            %y filter/.code={\pgfmathparse{#1/100}\pgfmathresult},
        ]


			\foreach \qec in {1, 2, 7, 10, 20}{
            \addplot+ table [x =timestamp, y =dist, smooth, each nth point=5]{results/waterfallp11nbQec\qec window1.csv};
        }
        	\legend{1, 2, 7, 10, 20}

    \end{axis}
\end{tikzpicture}
\begin{tikzpicture}
    \begin{axis}[
            ylabel={QoE},
            xlabel={time (s)},
            width=1.05\linewidth,
            height=0.5\linewidth,
            %x=2cm,
%    	    xtick={-10,-5,0,5,...,50},
            %ytick={0,0.1,0.2,...,1},
            %x tick label as interval,
            %xticklabels={%
            %    {500~ms},%
            %    {1000~ms},%
            %    {1500~ms},%
            %    {2000~ms},%
            %},
            cycle list name={My color list},
            legend cell align=left,
            xmin = 0,
            %ymin = 0,
            xmax = 10,
            %ymax = 1,
            ymajorgrids,
            %y filter/.code={\pgfmathparse{#1/100}\pgfmathresult},
        ]


			\foreach \qec in {1, 2, 7, 10, 20}{
            \addplot+ table [x =timestamp, y =dist, smooth, each nth point=5]{results/waterfallp11nbQec\qec window5.csv};
        }
        	\legend{1, 2, 7, 10, 20}

    \end{axis}
\end{tikzpicture}

    \caption{Instantaneous distance from the \ac{QEC} depending on the segment duration}
    \label{fig:instant_dist}
\end{figure}

\foreach \window in {1, 5} {
\begin{figure}

    \pgfplotscreateplotcyclelist{My color list}{%
    {black!25,solid, very thick},%
    {black!50,densely dashed, very thick},%
    {black!75,densely dotted, very thick},%
    {black!100,dash pattern=on 4pt off 1pt on 4pt off 4pt, very thick},%
    {black!25,dotted, very thick},%
}

\pgfplotsset{every axis legend/.append style={
        at={(0,0.97)},
anchor=south west,
draw=none,
fill=none,
legend columns=2,
column sep=15pt,
/tikz/every odd column/.append style={column sep=0cm},
%font=\tiny
}}

\pgfplotsset{grid style={dashed,gray}}
\pgfplotsset{minor grid style={dotted,red!20!gray}}
\pgfplotsset{major grid style={dotted,gray!50!black}}


\begin{tikzpicture}
    \begin{axis}[
            ylabel={QoE},
            xlabel={Timestamp},
            width=1.05\linewidth,
            height=1\linewidth,
            %x=2cm,
%    	    xtick={-10,-5,0,5,...,50},
            %ytick={0,0.1,0.2,...,1},
            %x tick label as interval,
            %xticklabels={%
            %    {500~ms},%
            %    {1000~ms},%
            %    {1500~ms},%
            %    {2000~ms},%
            %},
            cycle list name={My color list},
            legend cell align=left,
            xmin = 0,
            %ymin = 0,
            xmax = 10,
            ymax = 1,
            ymajorgrids,
            %y filter/.code={\pgfmathparse{#1/100}\pgfmathresult},
        ]


			\foreach \layout in {AverageEquiTiled, CubeMapLowerQuality, EquirectangularTiledLowerQuality, PyramidLowerQuality, RhombicDodecaLowerQuality}{
            \addplot+ table [x =timestamp, y =qoe ]{outputs2/bmxp01_00_expectedLiveQoE_\window s_quality\layout .csv};
        }
        	\legend{AverageEqui, CubeMapLower, EquirectangularTiledLower, PyramidLower, RhombicDodecaLower}

    \end{axis}
\end{tikzpicture}
\begin{tikzpicture}
    \begin{axis}[
            ylabel={CDF},
            xlabel={Orthodromic Distance},
            width=1.05\linewidth,
            height=1\linewidth,
            %x=2cm,
%    	    xtick={-10,-5,0,5,...,50},
            %ytick={0,0.1,0.2,...,1},
            %x tick label as interval,
            %xticklabels={%
            %    {500~ms},%
            %    {1000~ms},%
            %    {1500~ms},%
            %    {2000~ms},%
            %},
            cycle list name={My color list},
            legend cell align=left,
            xmin = 2,
            %ymin = 0,
            xmax = 3,
            ymax = 1,
            ymajorgrids,
            %y filter/.code={\pgfmathparse{#1/100}\pgfmathresult},
        ]


			\foreach \layout in {AverageEquiTiled, CubeMapLowerQuality, EquirectangularTiledLowerQuality, PyramidLowerQuality, RhombicDodecaLowerQuality}{
            \addplot+ table [x =timestamp, y =qoe ]{outputs2/bmxp01_00_expectedLiveQoE_\window s_quality\layout .csv};
        }
        	\legend{AverageEqui, CubeMapLower, EquirectangularTiledLower, PyramidLower, RhombicDodecaLower}

    \end{axis}
\end{tikzpicture}

    \caption{Instantaneous \ac{QoE} depending on the layout selected with a segment duration of \SI{\window}{\second}}
    \label{fig:instant_qoe_\window}
\end{figure}
}

% \begin{figure}
%     \input{plots/plot_instant_qoe_segmentSize.tex}
%     \caption{Instantaneous \ac{QoE} depending on the segment duration with the cube map layout}
%     \label{fig:instant_qoe_segmentSize}
% \end{figure}

In our second simulation we aggregated the results from the
head-position dataset and the results from the first simulation to
estimate the \ac{QoE} a real user would  observed using the different
layouts and using different segment sizes.
Figure~\ref{fig:instant_dist} shows the instantaneous variation of the
distance from the current \ac{QEC} depending on the video segment
duration. Figures~\ref{fig:instant_qoe_1} and \ref{fig:instant_qoe_5}
represent the instantaneous estimated \ac{MS-SSIM} depending on the
layout selected respectively for a video segment duration of
\SI{1}{\second} and \SI{5}{\second}. We observe that with a segment of
\SI{1}{\second}, the quality of the \ac{FoV} generated from the cube
map layout is pretty constant. The pyramid layout has a maximum
quality equal to the cube map layout but has abrupt quality variations
that are often not appreciated by the watchers. The rhombicDodeca and
the equirectangular tiled layout have a smooth \ac{QoE} variation but
have an average \ac{QoE} that is lower than the cube map layout.
Anyway they are almost always better than the default equirectangular
layout. With a \SI{5}{\second} video segment duration, the observation
are quite similar for the cube map and the rhombic dodeca layouts. The
average quality for the rhombic dodeca is now close the the quality of
the cube map layout. The quality variation for the equirectangular
tiled layout and the pyramid layout are even more brutal. Such a
quality variation will be noticed by the user. Moreover the quality of
those two layouts is often worse than the quality of the default
equirectangular video.
