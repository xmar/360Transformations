
%\documentclass[10pt,onecolumn,twoside]{IEEEtran} %!PN
%  \documentclass[10pt,twocolumn,twoside]{IEEEtran} %!PN
% \documentclass[12pt,onecolumn,twoside,draft]{IEEEtran} %!PN
%\documentclass[conference]{IEEEtran} %!PN
%\documentclass[conference]{IEEEtran}

\documentclass{sig-alternate}
%\documentclass{article}
%\usepackage{spconf} %ICME conf style

\usepackage{multirow}
\usepackage{amsfonts}
\usepackage{epsfig}
\usepackage{amsmath}
\usepackage{amssymb}
%%\usepackage[nolist]{acronym}
\usepackage[acronym]{glossaries}
\newcommand\acro[2]{\newacronym{#1}{#1}{#2}}
\newcommand\acroAlwaysShort[1]{\newglossaryentry{#1}{type=\acronymtype, name={#1}, description={#1}, text={#1}, first={#1}, plural={#1s}, firstplural={#1s}}}
\newcommand\acroShortSurname[2]{\newglossaryentry{#1}{type=\acronymtype, name={#2}, description={#2}, text={#2}, first={#2}, plural={#2s}, firstplural={#2s}}}
\newcommand\ac[1]{\gls{#1}}
\newcommand\acp[1]{\glspl{#1}}
\newcommand\acs[1]{\glsname{#1}}
\newcommand\acl[1]{\glsentrylong{#1}}

\acro{FoV}{Field of View}
\acro{ABR}{adaptive bit-rate}
\acro{DASH}{Dynamic Adaptive Streaming over HTTP}
\acro{CDN}{content delivery network}
\acro{CDF}{cumulative density function}
\acro{HMD}{Head-Mounted Display}
\acro{PDF}{probability density function}
\acro{RTP}{real-time protocol}
\acro{VR}{Virtual Reality}
\acro{QoE}{Quality of Experience}
\acro{VQM}{Video Quality Metric}
\acro{ILP}{integer linear program}
\acro{HDTV}{high definition television}
\acro{UGC}{User-Generated Content}
\acro{MPD}{Media Presentation Description}
\acro{PSNR}{Peak Signal Noise to Ratio}
\acro{GPU}{graphics processing unit}
\acro{CPU}{central processing unit}
\acro{MS-SSIM}{Multiscale - Structural Similarity}
\acro{API}{Application Programming Interface}
\acro{QEC}{Quality Emphasis Center}
\newglossaryentry{RoI}{type=\acronymtype, name={RoI}, description={RoI}, text={RoI}, first={Region of Interest~(RoI)}, plural={RoI}, firstplural={Regions of Interest~(RoI)}}
\acro{VM}{Virtual Machine}
\acro{SVC}{Scalable Video Coding}
\acro{GOP}{Group of Picture}
\acro{PMU}{Performance Monitoring Unit}
\acro{LAN}{Local Area Network}
\acro{AI}{Artificial Intelligence}
\acro{3D}{Three Dimentional}
\acro{OS}{Operating System}
\newglossaryentry{fps}{type=\acronymtype, name={fps}, description={frame per second}, text={frame per second}, first={frame per second~(fps)}, plural={fps}, firstplural={frames per second~(fps)}}
\newglossaryentry{p}{type=\acronymtype, name={p}, description={p}, text={~pixel}, first={~pixel (p)}, plural={p}, firstplural={~pixels (p)}}
\newglossaryentry{s}{type=\acronymtype, name={s}, description={s},
text={second}, first={~second~(s)}, plural={s}, firstplural={~seconds~(s)}}
\acroAlwaysShort{TCP}
\acroAlwaysShort{HTTP}
\acroAlwaysShort{MPEG}
\acro{RTT}{Round-Trip Time}
\acro{AVC}{Advanced Video Coding}
\acro{HEVC}{High Efficiency Video Coding}
\acro{ISO}{International Organization for Standardization}
%\acro{ISOBMFF}{\ac{ISO} base media file format}
\newglossaryentry{ISOBMFF}{type=\acronymtype, name={ISOBMFF}, description={International Organization for Standardization base media file format}, text={International Organization for Standardization base media file format}, first={International Organization for Standardization base media file format ISO/IEC 14496-12 (ISO BMFF)}, plural={ISO BMFFs}, firstplural={International Organization for Standardization base media file formats (ISO BMFFs)}}
\acro{MTU}{Maximum Transmission Unit}
\acro{AQM}{Active Queue Management}
\acro{I}{intra-predicted}
\acro{P}{inter-predicted}
\acro{B}{bidirectional}
\acro{VQMT}{Video Quality Measurement Tool}
\acro{EPFL}{Ecole Polytechnique F\'{e}d\'{e}rale de Lausanne}
\acroShortSurname{YUV}{YUV}
\acroShortSurname{RGB}{R'G'B'}
\acro{MSE}{Mean Square Error}
\acro{CMSE}{Commulative Mean Square Error}
 % acronymes + associated packages  are defined in the acronymes.tex file
\usepackage[english]{babel}
%\usepackage{cite}
\usepackage[numbers,sort]{natbib}
\renewcommand\citet[1]{\citeauthor{#1}~\cite{#1}} % redefine the \citet command to add a ~ space between authors and []
\usepackage{color}
\usepackage[dvipsnames]{xcolor}
\usepackage{stfloats}
%\usepackage{pst-gantt}
% \usepackage{algorithm}
% \usepackage{algorithmic}
\usepackage[linesnumbered,ruled,vlined,boxed,commentsnumbered]{algorithm2e}
\usepackage[noend]{algorithmic}
\usepackage{float}
\algsetup{linenosize=\tiny}
%\usepackage{caption}
%\usepackage{subcaption}
\usepackage[caption=false]{subfig}
\usepackage{pgfplots}
\usepackage{booktabs}
\usepackage{listings}
\usepackage[hidelinks]{hyperref}
%\usetikzlibrary{plotmarks}
\usetikzlibrary{shapes,positioning,3d,calc}
\usetikzlibrary{decorations,decorations.pathmorphing}
\usetikzlibrary{external}
\newcommand{\externaldirectory}{latex.out/}
\tikzexternalize[prefix=\externaldirectory]
\tikzexternalize % activate!
\tikzexternaldisable
\usepackage{wrapfig}
\usepackage{enumitem}
\usepackage{url}
\usepackage{etoolbox}

\lstset{%
  backgroundcolor=\color{gray!25},
  basicstyle=\sffamily \scriptsize,
  breaklines=true
}

%SI Units
\usepackage{siunitx}
\sisetup{detect-all}

%math tools
\usepackage{mathtools}
\usepackage{stmaryrd}
\usepackage{mathrsfs}
\usepackage{amssymb}
%Some math declarations
\DeclarePairedDelimiter{\ceil}{\lceil}{\rceil}
\DeclarePairedDelimiter{\parenthesis}{(}{)}
\DeclarePairedDelimiter{\set}{\{}{\}}
\DeclarePairedDelimiter{\norm}{|}{|}
\DeclarePairedDelimiter{\integerInterval}{\llbracket}{\rrbracket}
\DeclareMathOperator*{\minimize}{minimize}

% parag
\newcommand{\parag}[1]{\vspace{5pt}\noindent\textbf{#1}.\hspace{5pt}}

%Some symbols definitions
\newcommand{\packetset}{\mathcal P}
\newcommand{\frameset}{\mathcal F}
\newcommand{\packetsubset}{\mathcal P'}
\newcommand{\framesubset}{\mathcal F'}


% fix the issue of white character in acronym package
\usepackage{etoolbox}
\makeatletter
\patchcmd\@acf{\hskip\z@}{}{}{}
\patchcmd\@acf{\hskip\z@}{}{}{}
\makeatother


\usepackage{pgfplotstable}
\pgfplotsset{compat=1.8}


\newbool{NotesActivated}
\booltrue{NotesActivated}  %comment this line to remove all user comments

%include some macro specific for this paper
\newcommand\algoFontSize{8}
\newcommand{\noteGS}[1] {\ifbool{NotesActivated}{\color{red}\{\textbf{GS:}\textit{{#1}}\}\color{black}}{}}
\newcommand{\noteXC}[1] {\ifbool{NotesActivated}{\color{YellowOrange}\{\textbf{XC:}\textit{{#1}}\}\color{black}}{}}
\newcommand{\noteFB}[1] {\ifbool{NotesActivated}{\color{OliveGreen}\{\textbf{FB:}\textit{{#1}}\}\color{black}}{}}
\newcommand{\noteGA}[1] {\ifbool{NotesActivated}{\color{Blue}\{\textbf{GA:}\textit{{#1}}\}\color{black}}{}}
\newcommand{\noteGT}[1] {\ifbool{NotesActivated}{\color{Salmon}\{\textbf{GT:}\textit{{#1}}\}\color{black}}{}}
\newcommand\newpara[1]{\vspace{3pt}\noindent\textbf{#1}.\hspace{0.15cm}}
\newcommand\newsubpara[1]{\vspace{0.15cm}\noindent\textit{#1}.\hspace{0.15cm}}

%Define names of the Estimation Functions
\newcommand{\setPositive}{ % STYLE
   \text{\bf{\scriptsize+}}
}
\newcommand{\setNegative}{ % STYLE
   %\mathbb{\tiny-}
   \text{\large-}
}
\newcommand\constantParam[1]{
    {\scriptsize \{ #1 \}}
}


%\newcommand\R{\emph{Random}}
%\newcommand\TR{\emph{Type}}
%\newcommand\DR{\emph{Dependencies}}
%\newcommand\SP{\emph{DropSmall}}
%\newcommand\DSP{\emph{DepDropSmall}}
%\newcommand\DSM{\emph{DepDropBig}}
%\newcommand\DTSM{\emph{HybridDropBig}}
%\newcommand\DTSP{\emph{HybridDropSmall}}


\newcommand {\otoprule }{\midrule [\heavyrulewidth]}

\title{Geometric Layout for Navigable 360-Degree Video Delivery}

\tolerance=1
\emergencystretch=\maxdimen
\hyphenpenalty=10000
\hbadness=10000

\floatstyle{ruled}
\newfloat{ilp}{ht}{aux}
\floatname{ilp}{Integer Linear Program}


\newbool{doubleBlinded}
\booltrue{doubleBlinded}  %comment this line to remove all user comments

\makeatletter

\ifdoubleBlinded
\numberofauthors{1}
\else
\numberofauthors{3}
\fi

\author{ 
\alignauthor
\ifdoubleBlinded
        Paper ID 
\else
  Xavier Corbillon\\
  \affaddr{T\'{e}l\'{e}com Bretagne, IRISA, France}% \\
\alignauthor
  Alisa Devlic\\
  \affaddr{T\'{e}l\'{e}com Bretagne, IRISA, France}% \\
\alignauthor
  Gwendal Simon\\
  \affaddr{T\'{e}l\'{e}com Bretagne, IRISA, France}%\\
\fi
}



\makeatother

\usetikzlibrary{calc}
\usetikzlibrary{intersections}

%\definecolor{color1}{HTML}{73d216}
\definecolor{color1}{rgb}{0.66,0.76,0.23}
\definecolor{color2}{HTML}{c17d11}
\definecolor{color3}{HTML}{cc0000}
\definecolor{color4}{HTML}{204a87}
\definecolor{color5}{HTML}{ad7fa8}

\definecolor{tangoRed}{HTML}{cc0000}
\colorlet{titles}{color1}

\definecolor{midqualityOri}{HTML}{e9b96e}
%\definecolor{fullquality}{HTML}{8f5902}
\colorlet{midquality}{beamer@tbbrown!40!midqualityOri}
\colorlet{fullquality}{beamer@tbbrown}

\tikzset{
	spherical/.pic={
		\shade[ball color=midquality,opacity=0.40] (0,0) circle (#1 pt);
		\draw (-0.0352778*#1,0) arc (180:360:#1 pt and 0.5*#1 pt);
	    \draw[densely dashed] (-0.0352778*#1,0) arc (180:0:#1 pt and 0.5*#1 pt);
   	    \draw (0,0.0352778*#1) arc (90:270:0.5*#1 pt and #1 pt);
   	    \draw[densely dashed] (0,0.0352778*#1) arc (90:-90:0.5*#1 pt and #1 pt);
     	\draw (0,0) circle (#1 pt);
    }
}

\tikzset{
	pics/equirectangular/.style n args={3}{
		code={
			%mid-quality
			\draw[fill=midquality] (#2*0.00881945*#1-3*0.00881945*#1,#3*0.004961*#1-3*0.004961*#1)
			rectangle
			(#2*0.00881945*#1+4*0.00881945*#1,#3*0.004961*#1+4*0.004961*#1);

			%full-quality
			\draw[fill=fullquality] (#2*0.00881945*#1-0.00881945*#1,#3*0.004961*#1-0.004961*#1)
			rectangle
			(#2*0.00881945*#1+2*0.00881945*#1,#3*0.004961*#1+2*0.004961*#1);

			%grid
			\foreach \i in {-4,-3,-2,-1,0,1,2,3}{
				\foreach \j in {-4,-3,-2,-1,0,1,2,3}{
					\draw (\i*0.00881945*#1,\j*0.004961*#1) rectangle (\i*0.00881945*#1+0.00881945*#1,\j*0.004961*#1+0.004961*#1);
					}
				}
		}% code
	}% pic style
}%tikzset

\tikzset{
	cubemap/.pic={
		\draw[fill=midquality] (-0.0352778*#1,-0.006615*#1) rectangle (-0.017639*#1,0.006615*#1);
		\draw[fill=fullquality] (-0.017639*#1,-0.006615*#1) rectangle (0,0.006615*#1);
		\draw[fill=midquality] (0,-0.006615*#1) rectangle (0.017639*#1,0.006615*#1);
		\draw[fill=white] (0.017639*#1,-0.006615*#1) rectangle (0.0352778*#1,0.006615*#1);
		\draw[fill=midquality] (-0.017639*#1,0.006615*#1) rectangle (0,0.019844*#1);
		\draw[fill=midquality] (-0.017639*#1,-0.006615*#1) rectangle (0,-0.019844*#1);
%		\draw[draw=black,fill=none] (-0.0352778*#1,-0.006615*#1) rectangle (0.0352778*#1,0.006615*#1);
	}
}

\tikzset{
	pyramid/.pic={
		\draw[fill=fullquality] (-0.011759*#1,-0.006615*#1) rectangle (0.011759*#1,0.006615*#1);
		% triangle north
		\draw[fill=midquality] (-0.011759*#1,0.006615*#1)
			 -- (0.011759*#1,0.006615*#1)
			 -- (0,0.019844*#1)
			 -- cycle;
		% triangle south
		\draw[fill=midquality] (-0.011759*#1,-0.006615*#1)
			 -- (0.011759*#1,-0.006615*#1)
			 -- (0,-0.019844*#1)
			 -- cycle;
		% triangle west
		\draw[fill=midquality] (-0.011759*#1,-0.006615*#1)
			 -- (-0.011759*#1,0.006615*#1)
			 -- (-0.0352778*#1,0)
			 -- cycle;
		% triangle east
		\draw[fill=midquality] (0.011759*#1,-0.006615*#1)
			 -- (0.011759*#1,0.006615*#1)
			 -- (0.0352778*#1,0)
			 -- cycle;
	}
}

\tikzset{
	pics/losange/.style n args={3}{
		code={
			\draw[fill=#2, rotate around={#3:(0,0)}]
				(0,0)
				-- (#1, 0.75*#1)
				-- (2*#1, 0)
				-- (#1, -0.75*#1)
				--	cycle;
		}
	}
}

\tikzset{
	dodecahedron/.pic={
		\def\xshi{0}
		\pic at(\xshi,0) {losange={#1}{white}{0}};
		\pic at(\xshi,0) {losange={#1}{white}{287}};
		\pic at(\xshi+2*#1,0) {losange={#1}{midquality}{254}};
		\pic at(\xshi+1.46*#1,-1.93*#1) {losange={#1}{midquality}{0}};

		\def\xshi{2.89*#1}
		\pic at(\xshi,0) {losange={#1}{fullquality}{0}};
		\pic at(\xshi,0) {losange={#1}{fullquality}{287}};
		\pic at(\xshi+2*#1,0) {losange={#1}{midquality}{254}};
		\pic at(\xshi+1.46*#1,-1.93*#1) {losange={#1}{white}{0}};

		\def\xshi{-2.89*#1}%5.78
		\pic at(\xshi,0) {losange={#1}{midquality}{0}};
		\pic at(\xshi,0) {losange={#1}{midquality}{287}};
		\pic at(\xshi+2*#1,0) {losange={#1}{midquality}{254}};
		\pic at(\xshi+1.46*#1,-1.93*#1) {losange={#1}{white}{0}};
	}
}





\begin{document}
%\ninept

\maketitle

\begin{abstract}
Here is the Abstract.


\end{abstract}
%%% Local Variables:
%%% mode: latex
%%% TeX-master: "paper"
%%% End:


\section{Introduction}
\label{sec:introduction}

\subsection{Context and Motivations}

The popularity of interactive 360-degree video systems (also known as immersive omnidirectional video) 
has grown with the advent of both capturing systems
(including catadioptric optical systems and multi-camera with stitching systems) and video
consumption systems (including head-mounted display and interactive HTML5 video players).
However, the delivery of 360-degree video content, from the servers of the content providers
to the end-users,
is still a challenge. Head-mounted devices display two videos (one per
eye), each of them with a high resolution (typically $1080\times 1200$ for state-of-the-art devices)
and a high frame rate. Both videos, which correspond to the \ac{FoV} on both eyes, are extracted
from a wider spherical video with a resolution that is three to four times larger.

None of the current solutions for the delivery of 360-degree videos is entirely satisfactory. Sending only 
the \ac{FoV} videos is the least bandwidth-hungry implementation. It requires however the server to 
compute the \ac{FoV} from the spherical video for each end-user. Moreover, it does not enable fast
navigation within the scene: When the client changes the 
\ac{FoV}, the device cannot immediately display any video because it does not have information on the other
parts of the full spherical video. The device has to notify the server and to wait for the reception of the 
newly adjusted \ac{FoV} videos. Another delivery implementation is to send the full spherical video 
and to let the device
extract the \ac{FoV} videos. This solution enables fast navigation but the bandwidth requirements are 
significant.

We explore in this paper a solution where the server offers multiple \emph{versions} of the same 
360-degree  video. Each version is characterized by an \emph{angle of vision}. It contains the full spherical 
video with an emphasis on the given angle of vision, \textit{i.e.},
the part of the video that is in front of the angle of vision is at the highest quality and the video quality 
degrades for the other parts. The client device chooses the video version according to the eye attention 
such that the 
the front view of the video version is close to the \ac{FoV} videos. Thus, the \ac{FoV} extraction is done
from the high quality spherical video. When the end-user changes the 
\ac{FoV}, the device can always extract a \ac{FoV} video (sometimes from a low quality part of the 
full video) until it switches to a new video version that better suits the new visual attention.

\subsection{Limitations of Previous Work}

The principle of differentiated video quality in the delivery of omnidirectional videos has been sketched 
in a recent short paper by~\citet{ochi_live_2015}. Their proposal is based on the idea of video
tiling, which has been implemented for navigable panorama 
video~\cite{sanchez_compressed_2015,wang_mixing_2014,gaddam_tiling_2015}: The panorama video 
is cut into independent \emph{tiles} (typically $8\times 8$) and the server offers several
video qualities for each tile. The client selects the quality of each tile according to the eye attention. This
solution has the same advantages of our proposal, since it reduces the overall bandwidth by emphasizing
the video quality only for the parts that are actually watched, at the price of transient lower quality just after
brutal navigation events. However, this solution does not take into account the characteristics of 360-degree
videos. A spherical video can be mapped into an \emph{equirectangular} video without losing any information
for the extraction of \ac{FoV}, but the pixels at the poles are over-sampled (the number of pixels
used in the rectangle to depict a pixel into the sphere is higher at the pole than in the equator). 
This characteristic is known to degrade the
performance of traditional video encoders~\cite{wojciechowski_h.264_2006,yu_framework_2015}. It also 
makes equirectangular tiling less relevant.

The mapping of spherical videos into 2D video is a projection that generates distortion on the resulting
full 2D video, but some projections enable \ac{FoV} extraction without information loss. It is typically the case of 
projections on \emph{cube map}~\cite{Ng2005} and 
\emph{rhombic dodecahedron}~\cite{fu_rhombic_2009}. The previous work regarding mapping into these
geometrical layouts has focused on enabling efficient implementation of signal processing 
functions~\cite{kazhdan_metric-aware_2010} and improving the video encoding~\cite{tosic_low_2009}. 
However, to the best of our knowledge, the 
question of differentiated quality on the different faces of the layouts has not been studied so far.

Finally, a major content provider of 360-degree videos has recently released details about the 
implementation of its delivery platform~\cite{facebook}. The depicted system is based on the same
main idea as our proposal, where up to 30 versions of the same video are available according to angles
of vision. This description corroborates that, from an industrial perspective, the extra-cost of
generating and storing multiple versions of the same video is relevant with respect to the bandwidth
savings. The authors use a mapping of the spherical videos into a pyramid where the base is the front
view and the peak is behind. Yet, this mapping under-samples some pixels, which results in 
information loss and distorted extraction of \ac{FoV} videos.

\subsection{Our Contributions}

In this paper, we present a system for navigable 360-degree video delivery. Our focus is
about the mapping of spherical videos into geometrical layouts with differentiated qualities
with respect of angle of visions. Our contribution is twofold:
\begin{itemize}
\item We introduce a tool (released on open source in a public repository\footnote{url is hidden for blind
preservation.}) that enables the mapping from a spherical video
into any geometrical layout (and vice versa). The tool has two main features: it enables differentiated
quality for different parts of the 2D video with respect to the geometrical layouts, and it extracts the 
\ac{FoV} videos from any point in the sphere. With this tool, the scientific community will be able to 
study the mapping of spherical videos for the delivery of 360-videos and the implementation of delivery 
solutions.
\item We provide a comprehensive comparison of the different quality-differentiated 
mappings regarding
the bit-rate of the resulting videos, the quality of the extracted \ac{FoV} videos, and the capacity to
implement smooth navigation. Thus, we address most of
the questions raised by~\citet{facebook}: which layout provides the best trade-off between bit-rate and
quality, how many video versions should be offered by the server, and what tile quality implementation 
enables the smoothest navigation.
\end{itemize}

In this short paper, we restrict our study to essentials, and we provide only our main findings based on a 
sample of the results. \textit{Here are our findings...}




%%% Local Variables:
%%% mode: latex
%%% TeX-master: "paper"
%%% End:


% =============
\section{Background}

We now provide the background for our study.
First, we depict the overall architecture of the delivery system.
Then, we recall the main geometric layouts for spherical videos.

\subsection{Navigable 360-degree Video Delivery}

The principles of a navigable video delivery system are similar as in adaptive bit-rate
video systems such as \ac{DASH}. The server offers multiple versions of the same video
and the client
selects the most appropriate version according to some criteria. These versions
are cut into second-long segments such that the client can regularly switch from one
version to another. \AD{If the client moves the head before receiving a new video version, 
he/she will be displayed a new FoV video with gradually degrading quality that is adjusted to the head movement (as described in Section \ref{sec:context}). Exactly how many seconds should pass until receiving a new video version depends on the video content, the viewer's head movements (i.e., whether he/she watches a football match, news, or plays a computer game), and the experienced quality degradations variations, which needs to be investigated in the future work.}{}

In the case of wide video with different spatial qualities, the main idea is to spatially cut
the video into \emph{tiles}.
Then, two implementations are possible for the delivery system. \AD{First option is that}{} the server
offers each tile at different qualities. The client selects each tile version independently
and it has to reconstruct
the full video from these tiles before the \ac{FoV} extraction. This solution allows a
fine setting of qualities but
most of the computation is done at the client side. The second option,
which is the one we consider in this paper and is also the industrial implementation described
by~\citet{facebook}, is that
the server prepares $x$ versions related to $x$ different \acp{QEC}. Each version
is an arrangement of tile
qualities such that the tiles that are close to the \ac{QEC} are at high-quality
and the other tiles
are at a lower quality. The main advantages include an easy management of the server
(\textit{e.g.} small \emph{manifest} file), a simple selection process for the client (by
a distance computation), and no need of re-constructing the video before the \ac{FoV} extraction.

At the client side, the end-user moves its head to decide the \ac{FoV}. The head movements
are \AD{forward and backwards, side to side, and should to shoulder, referred to as \emph{pitch}, \emph{yaw}, and \emph{roll}, respectively.}{} 
 The center of the \ac{FoV} is a
point on the sphere, the size of the \ac{FoV} depends on the device (typically
around 100$^\circ$ in state-of-the-art devices), and the orientation of the extracted video
is related to the roll.

The complete analysis and evaluation of the navigable 360-degree video delivery system
is left for future work. Due to lack of space, we focus here on the geometric layout
of the video versions and the tile quality arrangement.


\subsection{Geometric Layouts for Spherical Videos}

The projection of a sphere into a plane (known as mapping) has been extensively studied
for centuries. In this paper, we consider the four projections that are the most natural
candidates for 360-degree video delivery. These layouts are depicted in Figure~\ref{fig:mapping}.

\begin{figure}[ht]
\centering
\begin{tikzpicture}
\def\sizeSphere{20}%pt
\def\ecartY{-1.2}%cm
\def\ecartX{6}

% da sphere
\pic [local bounding box=spher]  at (0,0) {spherical=15};

% recantagular
\pic [local bounding box=equi] at (-3,\ecartY) {equirectangular={\sizeSphere}{-1}{0}};

% cupe map
\pic [local bounding box=cubemap] at (-1,\ecartY) {cubemap=\sizeSphere};

% pyramid
\pic [local bounding box=pyra] at (1,\ecartY) {pyramid=\sizeSphere};

% rhombic
%\pgfdeclareimage[width=36 pt]{dodecahedron}{RhombicDodecahedron.png}
%\node at (3,\ecartY) (dodec)
%    {\pgfbox[center,center]{\pgfuseimage{dodecahedron}}};

\def\unitused{0.22}

\pic [local bounding box=dodeca] at (3,0.88*\ecartY) {dodecahedron=\unitused};

% links
\draw[-latex] (spher.180) -| (equi);
\draw[-latex] (spher.200) -| (cubemap);
\draw[-latex] (spher.340) -| (pyra);
\draw[-latex] (spher) -| (dodeca);

\node[font=\scriptsize,anchor=north] at (equi.south) {equirectangular};
\node[font=\scriptsize,anchor=north] at (cubemap.south) {cube map};
\node[font=\scriptsize,anchor=north] at (pyra.south) {pyramid};
\node[font=\scriptsize,anchor=north] at (dodeca.south) {\vphantom{y}dodecahedron};

\end{tikzpicture}
\caption{Projections into four geometric layouts}\label{fig:mapping}
\end{figure}

The advantages and shortcomings of every projection have been studied in the literature. From
the images that are
projected on an equirectangular, a cube map, and a rhombic dodecahedron, it is possible
to generate a \ac{FoV}
for any position and angle in the sphere. Indeed, no information is loss, because no pixel is under-sampled in
this projection (no pair of pixels in the sphere is projected on these geometrical
layouts in only one pixel). However,
some pixels from the spherical image are over-sampled in the projected image. It is typically the case for
the equirectangular layout, for which the projection generates a significant over-sampling at the poles. On the
contrary, the pyramid is not a geometric layout for lossless projections. Some pixels (those who are in the back
of the view face) are under-sampled, so two pixels can be projected into one pixel by interpolating their color
values. A \ac{FoV} that is extracted for positions near the back can suffer from distortion. \AD{Hence, this is the least probable head orientation of the video viewer.}{} 

\section{Geometric Layout Evaluation}

Our objective is to identify the layout that offers the best performance regarding the quality of 
the extracted\ac{FoV} video and the bit-rate of the delivered 360-degree video. We first introduce 
the tool that we release to generate and exploit quality-differentiated
360-degree videos on geometric layouts. Then, we describe the testbed that we set up to evaluate the
performance of the four main geometric layouts. Finally, we show the first results that we obtained.

\subsection{Our Tool to Project 360-degree Video Into Quality-Differentiated Layouts}

We do not enter into details. The code is available at \textit{hidden url}. The main features of this tool include:
\begin{itemize}
\item Projection from a spherical video into any of the four geometric layouts and vice versa. Our tool is in 
particular able to re-project the 360-degree videos that are encoded and stored in the equirectangular layout,
which corresponds to a large fraction of current catalogs of 360-degree videos. 
\item Adjusting video qualities for each geometric face of any layout by setting \noteGS{to be written}
\item \ac{FoV} extraction for any spherical coordinate and any angle, in particular for quality-differentiated videos.
\end{itemize}

The software can thus be used by the scientific community to study geometric layouts, quality arrangement over the tiles, and \ac{FoV} extraction strategies.

\subsection{Testbed Description}

The testbed that we
set up is depicted in Figure~\ref{fig:testbed}. From the original spherical video, we considered the 
projection 
into the four considered geometric layouts (equirectangular, cube map, pyramid, dodecahedron). For 
each layout, we built
$x$ different versions of the video, which corresponds to $x$ different \acp{QEC}. Then we simulated
a client watching in a specific direction. We selected, for each layout, the version such that the \ac{QEC}
is the closest to the given direction. We extracted the \ac{FoV} from this version. Finally, we compared
the quality of the four different \ac{FoV} with respect to the reference video that has been generated
without quality loss from the spherical video.

\begin{figure}[t]
\begin{tikzpicture}

\def\ecartx{1}
\def\ecartybig{0.5}
\def\ecartysmall{0.3}
\def\heimax{8*\ecartysmall+3*\ecartybig}
\def\sizeIcon{12}
\def\unitdodeca{0.11}

\pic [local bounding box = equiFirst] at (\ecartx,0.5*\heimax) 
		{equirectangular={\sizeIcon}{-1}{0}};
\pic [local bounding box = equiSecond, 
		below= \ecartysmall of equiFirst] {equirectangular={\sizeIcon}{0}{-1}};
%\pic [local bounding box = equiThird, 
%		below=\ecartysmall of equiSecond] {equirectangular={\sizeIcon}{0}{0}};

\pic [local bounding box = cubeFirst, below=\ecartybig of equiSecond] {cubemap=\sizeIcon};
\pic [local bounding box = cubeSecond, 
		below=\ecartysmall of cubeFirst] {cubemap=\sizeIcon};
%\pic [local bounding box = cubeThird, 
%			below=\ecartysmall of cubeSecond] {cubemap=\sizeIcon};

\pic [local bounding box = pyraFirst, below=\ecartybig of cubeSecond] {pyramid=\sizeIcon};
\pic [local bounding box = pyraSecond, 
		below=\ecartysmall of pyraFirst] {pyramid=\sizeIcon};
%\pic [local bounding box = pyraThird, 
%			below=\ecartysmall of pyraSecond] {pyramid=\sizeIcon};

\pic [local bounding box = dodecaFirst, below=\ecartybig of pyraSecond, 
		xshift=-3pt] {dodecahedron=\unitdodeca};
\pic [local bounding box = dodecaSecond, below=\ecartysmall of dodecaFirst, xshift=-6pt,
		yshift=2pt] {dodecahedron=\unitdodeca};
%\pic [local bounding box = dodecaThird, 
%			below=\ecartysmall of dodecaSecond] {dodecahedron=\unitdodeca};

% ==== sphere
\pic [local bounding box=spher, yshift=10pt,
		right=-\ecartx cm of pyraFirst] at (0,0) {spherical=10};
		
% == links
\draw[-latex] (spher) to ([xshift=-3pt, yshift=5pt]equiSecond.west);
\draw[-latex] (spher) to ([xshift=-3pt, yshift=5pt]cubeSecond.west);
\draw[-latex] (spher) to ([xshift=-3pt, yshift=5pt]pyraSecond.west);
\draw[-latex] (spher) to ([xshift=-3pt, yshift=5pt]dodecaSecond.west);

% ==== selected versions
\pic [local bounding box = chosenEqui, 
		right= \ecartx of equiFirst] {equirectangular={\sizeIcon}{-1}{0}};
\pic [local bounding box = chosenCube, right=\ecartx of cubeFirst] {cubemap=\sizeIcon};
\pic [local bounding box = chosenPyra, 
		right=\ecartx of pyraSecond] {pyramid=\sizeIcon};
\pic [local bounding box = chosenDodeca, right=\ecartx of dodecaFirst, 
		xshift=-5pt, yshift=3.5pt] {dodecahedron=\unitdodeca};
		
% == links
\draw[-latex] ([xshift=1.5pt]equiFirst.east) to ([xshift=-1pt]chosenEqui.west);
\draw[-latex] ([xshift=1.5pt]cubeFirst.east) to ([xshift=-1pt]chosenCube.west);
\draw[-latex] ([xshift=1.5pt]pyraSecond.east) to ([xshift=-1pt]chosenPyra.west);
\draw[-latex] ([xshift=0.5pt]dodecaFirst.east) to ([xshift=-1pt]chosenDodeca.west);

% == legend
\tikzset{
	legendNode/.style={
		font=\scriptsize,
		text width=1cm,
		align=center,
		anchor=south
	}
}
\node[legendNode] at (chosenEqui |- equiFirst.north) {selected version};
\node[legendNode] at (equiFirst.north) {offered versions};

% ==== fov

\tikzset{
	pics/equirec/.style n args={4}{
		code={
			\draw[draw=none,fill=gray!30] (-0.0352778*#1, -0.019844*#1) rectangle (0.0352778*#1, 0.019844*#1);
			\draw[draw=none,fill=gray!70] (0.0088194*#2*#1-#4*0.0088194*#1, 0.0066147*#3*#1 - #4*0.0066147*#1) rectangle (0.0088194*#2*#1 + #4*0.0088194*#1, 0.0066147*#3*#1 + #4*0.0066147*#1);
			\draw[draw, very thick, densely dotted,fill=none] (-0.0352778*#1, -0.019844*#1) rectangle (0.0352778*#1, 0.019844*#1);
		}		
	}
}

\pic [local bounding box=fovEqui, right=\ecartx of chosenEqui] {equirec={\sizeIcon}{-2}{-1}{2}};
\pic [local bounding box=fovCube, right=\ecartx of chosenCube] {equirec={\sizeIcon}{1}{0}{3}};
\pic [local bounding box=fovPyra, right=\ecartx of chosenPyra] {equirec={\sizeIcon}{0}{-1}{2}};
\pic [local bounding box=fovDodeca] at (fovPyra |- chosenDodeca) {equirec={\sizeIcon}{0}{0}{3}};

\node[legendNode] at (fovEqui |- equiFirst.north) {\ac{FoV} extraction};

% == links

\draw[-latex] ([xshift=1.5pt]chosenEqui.east) to ([xshift=-1pt]fovEqui.west);
\draw[-latex] ([xshift=1.5pt]chosenCube.east) to ([xshift=-1pt]fovCube.west);
\draw[-latex] ([xshift=1.5pt]chosenPyra.east) to ([xshift=-1pt]fovPyra.west);
\draw[-latex] ([xshift=0.5pt]chosenDodeca.east) to ([xshift=-1pt]fovDodeca.west);

% ==== reference video
\tikzset{
	refVideo/.pic={
		\draw[draw=black, very thick, densely dotted, fill=gray!70] (-0.0352778*#1, -0.019844*#1) rectangle (0.0352778*#1, 0.019844*#1);
	}
}

\pic[local bounding box=fovRef,yshift=-15pt] 
		at (fovDodeca |- dodecaSecond.south) {refVideo=\sizeIcon};
\node[legendNode, text width=2cm] at (fovRef.north) {reference video};
		
\draw[-latex] (spher.south) |- (fovRef.west);

% ==== comparison MS-SSIM

\node[rounded corners,rectangle,thick,fill=gray!10,
		text width=1.5cm,align=center,anchor=west] (msssim)
		at ([xshift=\ecartx cm]fovEqui |- spher) {MS-SSIM comparison};
		
% == links

\draw[-latex] (fovEqui.east) -| (msssim.45);
\draw[-latex] (fovCube.east) -| (msssim.135);
\draw[-latex] (fovPyra.east) -| (msssim.225);
\draw[-latex] (fovDodeca.east) -| (msssim.270);
\draw[-latex] (fovRef.east) -| (msssim.315);

\end{tikzpicture}

\caption{Our testbed: the spherical video is projected into four geometric layouts with variable qualities and
one layout at full-quality for reference. The best version is selected, and the \ac{FoV} is extracted. The quality
of the extracted \acp{FoV} are compared with respect to the reference video.}\label{fig:testbed}
\end{figure}

To generate the different videos, we made some choices, which are mostly in conformance to the
literature and to real implementation of 360-degree video delivery systems. Our future work 
includes to study in more details the impact of some of these choices on
the overall performance of the system; it is not our objective here.

We generated $x=16$ different video versions for each layout. This number of versions is smaller than
the number of videos (with pyramid layouts) given by~\citet{facebook}. It is however closer to the
number of video representations that are recommended in rate-adaptive streaming 
systems~\cite{Aparicio-PardoP15}. We describe now specific quality arrangements per layout:
%\begin{description}
%\item[equirectangular] 

\parag{Equirectangular}We cut the equirectangular layout into $8\time 8$ tiles as proposed in
the literature related to panorama video~\cite{gaddam_tiling_2015}. Tiles are thus rectangular.
We considered three different qualities: the \emph{full quality}, which corresponds to the quality of the
input spherical video, a \emph{medium quality}, which is set as half as good as the full quality, and a 
\emph{low quality}, which is one quarter of the full quality. The quality arrangement is done such that
the $3\time 3$ tiles that are around the \ac{QEC} are full quality, the $7\time 7$ tiles around 
the \ac{QEC} (excluding the closest $3\time 3$) are with the medium quality, and the remaining tiles
are with the low quality.

\parag{Cube Map}To generate the $x=16$ \acp{QEC}, we rotate the cube in the sphere so that the
center of the square front face is the \ac{QEC}. The front face is at full quality, while the left, right, 
top, and bottom faces are mid quality and the back face is low quality.

\parag{Pyramid}This layout differs from other layouts. First, it does not preserve
the pixel information of the spherical videos. Second, the distortion depends on the size of the square
base face and to the distance from the peak. As for the cube map, we rotate the pyramid to adjust
the center of the base face to the \ac{QEC}. The base is at full quality while the other faces are
at a medium quality. \noteGS{To be written}

\parag{Rhombic Dodecahedron}We rotated the dodecahedron such that the \ac{QEC} is between
the two same faces (say face\,1 and face\,2). Both faces are at full quality. Then, the eight 
faces that are around
these faces are at medium quality, and the two remaining faces are at low quality.

We observed that, for the quality setting that we decided for the equirectangular (medium quality and 
low quality at half, respectively a quarter, quality of the full quality), the bit-rate of the
generated equirectangular layouts are greater than for those of the other layouts. To get the
same bit-rate for each video version, we apply the following process. We set that, for a given 
360-degree video, the projection on the equirectangular provides the \emph{bit-rate budget}. For the
other layout, the low quality is kept at a quarter of the full quality, but we explore the range of
possible medium qualities until the bit-rate of the generated video version is equal to the 
bit-rate budget.

\subsection{Results}

\section{Discussion and Conclusion}

\newpage
%%%%%%%%%%%%%%%%%%%%%%%%%%%%%%%
%\bibliographystyle{IEEEtran}
%\bibliographystyle{IEEEbib}
\bibliographystyle{abbrvnat}  
%\bibliographystyle{abbrv}  
\bibliography{biblio}

\end{document}

